
\section*{Methods}


\subsection*{Model overview}

We used a simple model of competition to evaluate how evolving investment in 
competition affects coexistence among species.
Each of $n$ species has $q$ competition axes.
Competition axes are combinations of multiple traits that together affect 
competition similarly (Figure \ref{fig:model-description}).
Investing in an axis has a cost to the population growth rate and 
a benefit via reduced interspecific competition.
The overall competition experienced by each species is a product of 
the axes in all species in the community.
This is because all species have at least two axes, 
one conflicting axis, and one ameliorative axis.
For a conflicting competition axis, investment by species $i$
strengthens the competitive effect that species $i$ has on all
other competitors.
For an ameliorative axis, all species' competition with 
$i$ is reduced when it invests in this axis.




All species are symmetrical in that when species share the same axis 
values, they will always have the same per-capita effect on each other.
In both the costs and benefits, axis effects are concave functions that,
combined, ensure that one fitness peak exists for each axis.
Axes can also have non-additive trade-offs that either increase or decrease
the cost associated with increasing multiple axes compared to increasing
just one \citep{Northfield2021}.

% Here, I think it makes sense to define what we mean by axes:
We could expect conflicting evolution to occur when contest competition
leads to arms races among competitors
\citep{Abrams1994}.
Competitor evolution might be ameliorative when they evolve
dissimilar resource-usage traits to reduce competition \citep{Roughgarden1976}.


We used a discrete-time, modified Lotka--Volterra competition model similar to
that by \citet{Northfield2013a}.
In it, species $i$ has a length-$q$ vector of axes ($\mathbf{v}_i$), and
its per-capita growth---equivalent to fitness---is

\begin{equation} \label{eq:fitness}
    F_{i} = \exp \left\{ r_i(\mathbf{v}_i) - 
        \alpha_{0} \, N_i - \sum_{j \ne i}^{n}{
            \alpha_{ij}(\mathbf{v}_i, \mathbf{v}_j) \, N_j}  
    \right\}\textrm{,}
\end{equation}

\noindent where $N_i$ is the population density of species $i$.
The parameter $r(\mathbf{v}_i)$ describes how axes affect
the growth rate:

\begin{equation} \label{eq:growth-rate}
\begin{split}
    r(\mathbf{v}_i) &= r_0 - f \, \mathbf{v}_i^{\textrm{T}} \, \mathbf{C} ~ \mathbf{v}_{i} \\
    \mathbf{C} &= \begin{pmatrix}
        1         & \ldots & \eta_{1q} \\
        \vdots    & \ddots & \vdots \\
        \eta_{q1} & \ldots & 1      \\
        \end{pmatrix}
    \textrm{,}
\end{split}
\end{equation}

\noindent where $r_0$ is the baseline growth rate,
$f$ is the cost of increasing axes on the growth rate, and
$\eta_{k,l}$ is the non-additive trade-off of increasing both the
$k$\textsuperscript{th} and $l$\textsuperscript{th} axes.
When $\eta > 0$, increasing multiple axes incurs an extra cost.
Non-additive trade-offs are symmetrical (i.e., $\eta_{k,l} = \eta_{l,k}$ for all
$l$ and $k$), and all values on the diagonal of $\mathbf{C}$ are 1.


The term $\alpha_{ij}(\mathbf{v}_i, \mathbf{v}_j)$
in equation \ref{eq:fitness} represents how axes influence the effects
of interspecific competition:

\begin{equation} \label{eq:competition}
\begin{split}
    \alpha_{ij}(\mathbf{v}_i, \mathbf{v}_j) &= \alpha_0 ~\exp \left\{
        - \mathbf{v}_i^{\textrm{T}} \mathbf{v}_i -
        \mathbf{v}_j^{\textrm{T}} \mathbf{D} \mathbf{v}_j \right\} \\
    \mathbf{D} &= \begin{pmatrix}
        d_1     & \ldots    & 0 \\
        \vdots  & \ddots    & \vdots \\
        0       & \ldots    & d_q
        \end{pmatrix}
	\textrm{,}
\end{split}
\end{equation}



\noindent where $\alpha_0$ is the base density dependence.
Matrix $\mathbf{D}$ contains parameters that determine how evolution of axes
in one species affects competition experienced by others:
When $d_k < 0$, investment by species $i$ in axis $k$ decreases the
effect of competition on species $i$, but increases it in all others
(i.e., axis $k$ is conflicting).
Alternatively, when $d_k > 0$, the same investment decreases the effect of
competition on all species (i.e., axis $k$ is ameliorative)
\citep{Northfield2013a}.


The relationships between axis values and other components of fitness
(growth rates and effects of competition) are of the form
$X \propto v^2$ for parameter $X$, so $v = z$ is equivalent to $v = -z$.
To avoid alternative outcomes due to artifacts of this relationship,
axes are not allowed to be $< 0$.
We did this by passing the equation for $\mathbf{v}_{t+1}$ through a
ramp function.
We used a ramp function instead of absolute values
because the latter causes fluctuations
in the axis values when they approach zero (they ``bounce off''
the zero bound) that persist for a very long time;
this caused the simulations to take a prohibitively long time to reach
equilibrium.
A more important disadvantage is that $d \lvert x \rvert / dx$ is
undefined when $x = 0$.
This implementation and its consequences on resulting derivatives are in
Appendix A.

Species started with an abundance of 1.
We tracked densities through time and considered a species extinct if its 
density fell below 1.



% \subsection*{Adaptive dynamics}
%
% We started simulations with a single competitive species with axis values set to zero.
% We tracked species population densities through time using equation \ref{eq:fitness} and
% considered a species extinct if its density fell below $10^{-4}$.
% Species produced daughter species with a probability of 0.01 per species per time step.
% We generated daughter-species axis values from normal distributions with means of the
% mother axis values and standard deviations of $\sigma_{d}$.


\subsection*{Quantitative genetics}

We used a quantitative genetics framework for axis evolution.
We assumed that all axes in $\mathbf{v}_i$ represent means for species $i$
and that their among-individual distributions are symmetrical with additive
genetic variance $\sigma^2_A$.
Assuming also that $\sigma^2_A$ is relatively small
\citep{Iwasa1991a,Abrams2001a,Abrams1993b}, 
axes at time $t+1$ are

\begin{equation} \label{eq:axis-change}
    \mathbf{v}_{i,t+1} = \mathbf{v}_{i,t} + \left( \frac{1}{F_i}
        \frac{\partial F_i}{\partial \mathbf{v}_{i,t}} \right) \sigma^2_A
    \textrm{.}
\end{equation}

To determine the stability of ending points (axis values and abundances of
surviving competitor(s)), we computed the $n (q+1) \times n (q+1)$ Jacobian matrices
of first derivatives for the axes and abundances of each species (Equation \ref{eq:jacobian}).
We then computed the primary eigenvalue of this matrix ($\lambda$).
We considered a state stable when $\lambda < 1$,
neutrally stable when $\lambda = 1$,
and unstable when $\lambda > 1$.

Full analytical solutions to matrix derivatives (for axis change and
Jacobian matrices) are found in appendix A.
We also analyzed equilibrium solutions for the 2 axis case.
This is found in Appendix B.


\subsection*{Stochasticity}

We added stochasticity to axis evolution by creating phenotypes
($\mathbf{\ddot{v}}$) that are the product of the
genotypes ($\mathbf{v}$) and a log-normal error term:

\begin{equation} \label{eq:V-stochasticity}
\begin{split}
    \mathbf{\ddot{v}}_{i,t+1} &= \mathbf{v}_{i,t+1} \; \text{e}^{\varepsilon_v} \\
    \varepsilon_v &\sim \text{N}(0, \, \sigma^2_V)
    \text{.}
\end{split}
\end{equation}

Because the phenotypes interact with the environment, they are used
to calculate all species' fitness values.
Genotypes change through time based on the previous time point's 
genotypes and how the phenotypes affect fitness:

\begin{equation} \label{eq:axis-change-stochastic}
    \mathbf{v}_{i,t+1} = \mathbf{v}_{i,t} + \left( \frac{1}{F_i}
        \frac{\partial F_i}{\partial \mathbf{\ddot{v}}_{i,t}} \right) \sigma^2_A
    \textrm{.}
\end{equation}





% \subsection*{Simulations}
% 
% 
% We started simulations with one species, and each of $n-1$ new species
% was added every 500 generations.
% Species started with an abundance of 1.
% We continued simulations for another 20,000 generations after all
% species were added.
% We tracked densities through time and considered a species extinct if its 
% density fell below 1.


\subsection*{Code}

We simulated models using a combination of R \citep{RCoreTeam2020} and
C++ via the Rcpp and RcppArmadillo packages
\citep{Eddelbuettel2014a,Eddelbuettel2013a,Sanderson2016}.
All code can be found on GitHub
(\url{https://github.com/lucasnell/sauron}).



