
\section*{Methods}

% ------------------------------------------------------------------------------

\subsection*{Model overview}



We used a discrete-time, modified Lotka--Volterra competition model similar to
that by \citet{Northfield2013a}.
In it, communities consist of $n$ total species, and 
species $i$ has investments in both conflicting ($x_i$) and 
partitioning ($p_i$) traits.
Its per-capita growth---here equivalent to fitness---is

\begin{equation} \label{eq:fitness}
    F_{i} = \exp\left\{
        r(x_i, p_i)
        - \alpha_0 N_{i} 
        - \sum_{j \ne i}^{n}{
            \alpha_{ij}(x_i, p_i, x_j, p_j) \, N_j}
        \right\}
    \textrm{,}
\end{equation}


\noindent where $N_i$ is the population density of species $i$,
and $\alpha_0$ is the base density dependence.
The parameter $r(x_i, p_i)$ describes the direct costs of investing to
the growth rate:

\begin{equation} \label{eq:growth-rate}
    r(x_i, p_i) = r_0 - f \, \left( x_i^2 + 2 \eta \, x_i \, p_i + p_i^2 \right)
    \textrm{.}
\end{equation}

\noindent Here, $r_0$ is the baseline growth rate,
$f$ is the base investment cost, and
$\eta$ is the non-additive cost for investing in both suites 
simultaneously.
When $\eta > 0$, costs are super-additive, and 
when $\eta < 0$, costs are sub-additive.


The term $\alpha_{ij}(x_i, p_i, x_j, p_j)$ in equation \ref{eq:fitness}
represents how investments influence the effects
of interspecific competition:

\begin{equation} \label{eq:competition}
    \alpha_{ij}(x_i, p_i, x_j, p_j) = \alpha_0 \,
        \text{e}^{-x_i^2 - p_i^2 + d_x \, x_j^2 - d_p \, p_j^2 }
	\textrm{,}
\end{equation}


\noindent where $d_x$ and $d_p$ determine how strongly investments in each 
suite of traits in one species affect competition experienced by others.
With increasing values of $d_x$, each unit of investment by species $i$ in 
conflicting traits has a greater positive effect on the competition 
experienced by others in the community from species $i$.
With increasing values of $d_p$, investment by species $i$ in partitioning
traits has a stronger negative effect on competition experience by all 
others in the community.
Values of $d_x$ and $d_p$ have no direct effects on the investing species.



In both the costs and benefits, investment effects are concave functions that,
combined, ensure that one fitness peak exists for each suite of traits.
We did this by making the relationships between investments and other 
components of fitness (growth rates and effects of competition) of the form
$Z \propto x^2$ and $Z \propto p^2$ for hypothetical parameter $Z$.
A side effect is that $Z(x) = Z(-x)$ and $Z(p) = Z(-p)$.
To avoid alternative outcomes due to artifacts of these relationships,
investments are not allowed to be $< 0$.
We did this by passing the equations for $x_{t+1}$ and $p_{t+1}$ (below) 
through a ramp function.
We used a ramp function instead of absolute values
because the latter causes fluctuations
in the investments when they approach zero (they ``bounce off''
the zero bound) that persist for a very long time;
this caused the simulations to take a prohibitively long time to reach
equilibrium.
A more important disadvantage is that $d \lvert z \rvert / dz$ is
undefined when $z = 0$.
This implementation and its consequences on resulting derivatives are in
Appendix A.






% \subsection*{Adaptive dynamics}
%
% We started simulations with a single competitive species with investments set to zero.
% We tracked species population densities through time using equation \ref{eq:fitness} and
% considered a species extinct if its density fell below $10^{-4}$.
% Species produced daughter species with a probability of 0.01 per species per time step.
% We generated daughter-species investments from normal distributions with means of the
% mother investments and standard deviations of $\sigma_{d}$.


\subsection*{Quantitative genetics}

We used a quantitative genetics framework for investment evolution.
We assumed that both $x_i$ and $p_i$ represent means for species $i$
and that their among-individual distributions are symmetrical with additive
genetic variance $\sigma^2_A$.
Assuming also that $\sigma^2_A$ is relatively small
\citep{Iwasa1991a,Abrams2001a,Abrams1993b}, 
investments at time $t+1$ are

\begin{equation} \label{eq:invest-change}
\begin{split}
    x_{i,t+1} &= x_{i,t} + \left( \frac{1}{F_i}
        \frac{\partial F_i}{\partial x_{i,t}} \right) \sigma^2_A \\
         &= x_{i,t} + 2 \, \sigma_A^2 \left[
            \alpha_0 \, \text{e}^{-x_{i,t}^2 - p_{i,t}^2} \, x_{i,t} \,
            \sum_{j \ne i}^{n}{ N_{j,t} \, \text{e}^{ 
                -d_x \, x_{j,t}^2 - d_p \, p_{j,t}^2 } }
            - f \left( x_{i,t} + \eta \, p_{i,t} \right) 
        \right]\\[1ex]
    p_{i,t+1} &= p_{i,t} + \left( \frac{1}{F_i}
        \frac{\partial F_i}{\partial p_{i,t}} \right) \sigma^2_A \\
        &= p_{i,t} + 2 \, \sigma_A^2 \left[
            \alpha_0 \, \text{e}^{-x_{i,t}^2 - p_{i,t}^2} \, p_{i,t} \,
            \sum_{j \ne i}^{n}{ N_{j,t} \, \text{e}^{ 
                -d_x \, x_{j,t}^2 - d_p \, p_{j,t}^2 } }
            - f \left( p_{i,t} + \eta \, x_{i,t} \right) 
        \right]
    \textrm{.}
\end{split}
\end{equation}


\subsection*{Stability}

To determine the stability of communities, we computed the Jacobian matrices
of first derivatives for the investments and abundances of each species
(Equation \ref{eq:jacobian}).
We then calculated the primary eigenvalue of this matrix ($\lambda$).
We considered a state stable when $\lambda < 1$,
neutrally stable when $\lambda = 1$,
and unstable when $\lambda > 1$.
All the derivatives for Jacobian matrices are listed in appendix B.




\subsection*{Stochasticity}

We added stochasticity to investment evolution to evaluate how coevolution
is affected by species being unable to perfectly evolve to fitness peaks.
Many mechanisms can constrain evolution such that phenotypes do not 
align with fitness peaks
\citep[e.g., behavioral syndromes, pleiotropy, time
lags;][]{Langerhans2002,Sih2004,Padilla1996}.
As indirect effects of stress responses, non-adaptive plastic
changes are probably ubiquitous in natural communities
\citep{Miner2005,Ghalambor2007}.
Moreover, they can affect adaptive evolution \citep{Ghalambor2015a}
and community dynamics \citep{Langerhans2002,Peacor2006}.
In our model, we added stochasticity via non-adaptive plasticity
by having phenotypes ($\ddot{x}$ and $\ddot{p}$) 
that are the product of the genotypes (${x}$ and ${p}$) and a log-normal
error term:
$\ddot{x}_{i,t+1} = x_{i,t+1} \; \text{e}^{\varepsilon_x}$ and
$\ddot{p}_{i,t+1} = p_{i,t+1} \; \text{e}^{\varepsilon_p}$,
where $\varepsilon_x$ and $\varepsilon_p$ are generated from normal
distributions with means of 0 and variances $\sigma^2_{x}$ and $\sigma^2_{p}$,
respectively.
Because the phenotypes interact with the environment, they are used
to calculate all species' fitness values.
Genotypes change through time based on the previous time point's 
genotypes and how the phenotypes affect fitness:

\begin{equation} \label{eq:invest-change-stochastic}
\begin{split}
    x_{i,t+1} &= x_{i,t} + \left( \frac{1}{F_i}
        \frac{\partial F_i}{\partial \ddot{x}_{i,t}} \right) \sigma^2_A \, \text{e}^{\varepsilon_x} \\
    p_{i,t+1} &= p_{i,t} + \left( \frac{1}{F_i}
        \frac{\partial F_i}{\partial \ddot{p}_{i,t}} \right) \sigma^2_A \, \text{e}^{\varepsilon_p}
    \textrm{.}
\end{split}
\end{equation}





\subsection*{Code}

We implemented this model through an R package \texttt{sauron}
that uses a combination of R \citep{RCoreTeam2020} and
C++, the latter via the Rcpp and RcppArmadillo packages
\citep{Eddelbuettel2014a,Eddelbuettel2013a,Sanderson2016}.
The package and its implementation for this paper can be found on GitHub
(\url{https://github.com/lucasnell/sauron}).



