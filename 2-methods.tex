
\section*{Methods}

\subsection*{Ecological dynamics}

We used a discrete-time, modified Lotka--Volterra competition model
similar to that used by Northfield and Ives
\citep{Northfield2013a}. Communities consist of $n$ species,
and each species $i$ can invest in both partitioning ($p_{i}$) and
conflict ($x_{i}$) traits. A species' per capita population growth
rate is

\begin{equation} \label{eq:pg-growth}
    F_{i} = \exp\left\{
        r(p_i, x_i)
        - \alpha_0 N_{i} 
        - \sum_{j \ne i}^{n}{
            \alpha_{ij}(p_i, x_i, p_j, x_j) \, N_j}
        \right\}
    \textrm{,}
\end{equation}

\noindent where $N_{i}$ is the population density of species $i$, and
$\alpha_{0}$ is the strength of intraspecific competition which is
fixed to the same value for all species and unchanged by trait
evolution. The function $r( p_{i},x_{i} )$ is the per
capita population growth rate of focal species $i$ in the absence of
any intra- or interspecific competition. This function describes the
direct fitness costs of investment as a reduction from the baseline per
capita population growth rate $r_{0}$:

\begin{equation} \label{eq:growth-rate}
    r(p_i, x_i) = r_0 - f \, \left( p_i^2 + 2 \eta \, p_i \, x_i + x_i^2 \right)
    \textrm{.}
\end{equation}

\noindent Here, $f$ is a coefficient giving the cost of investment in a single
trait type, and $\eta$ governs the non-additivity of costs for
investing in both suites simultaneously. When $\eta > 0$, costs are
super-additive, and when $\eta < 0$, costs are sub-additive.
Super-additive tradeoffs occur when investment in one suite of traits
makes investment in the other suite of traits more costly \citep{Garland2022}.
% 
This might occur, for example, if competition for light selects for greater
growth in plants, which then increases the cost of chemical defenses to
specialist herbivores.
% 
Sub-additive costs could be caused by shared
biochemical/physiological pathways that make investment in one type of
trait also increase the effects of the other type \citep{Garland1994,
Garland2022}. Sub-additive costs could also arise when
partitioning and conflict traits are correlated through an underlying
trait \citep{Strauss2004}, and therefore the evolution of one
type of trait reduces the cost of the other. For example, an increase in
body size could both improve the ability of a species in agonistic competition
and allow it to feed on larger prey, thereby simultaneously leading to 
resource partitioning.

The term $\alpha_{ij}( p_{i},x_{i},p_{j},x_{j} )$ gives the
pairwise interspecific competition coefficient, which is a function of
the investments in partitioning and conflict traits for species $i$
and $j$. We assume this has the form

\begin{equation} \label{eq:competition}
    \alpha_{ij}(p_i, x_i, p_j, x_j) = \alpha_0 \,
        \text{e}^{-p_i^2 - x_i^2 - d_p \, p_j^2 + d_x \, x_j^2 }
	\textrm{.}
\end{equation}


\noindent Interspecific competition is scaled by $\alpha_{0}$, such that the
effects of investment in partitioning and conflict traits are defined
relative to the fixed strength of intraspecific competition. The
parameters $d_{p}$ and $d_{x}$ correspond to partitioning and
conflict traits, respectively, and determine how strongly investments in
each suite of traits in one species affect competition experienced by
other species. With increasing values of $d_{p}$, investment by
species $i$ in partitioning traits has a stronger negative effect on
the interspecific competition experienced by other species. With
increasing values of $d_{x}$, investment by species $i$ in conflict
traits has a stronger positive effect on competition experienced by all
other species. Values of $d_{p}$ and $d_{x}$ only affect a focal
species $i$ through the competition it experiences from other species
$j \neq i$.

For both costs (eq. \ref{eq:growth-rate}) and 
benefits (eq. \ref{eq:competition}), the functions giving the
effects of trait investment on fitness are concave so that a single
fitness peak exists for each suite of traits. We did this because our
goal was not to model the diversification process, and because this made
the equations more mathematically tractable, preventing runaway
evolution to infinite trait values. Investments are constrained to be
non-negative by passing the equations for $p_{t + 1}$ and
$x_{t + 1}$ (eq. \ref{eq:invest-change} below) through a ramp function. 
Details are given in Appendix A.

\subsection*{Evolutionary dynamics}

We used a quantitative genetics framework for modeling coevolution of
traits by different species. We assumed that both $p_{i}$ and
$x_{i}$ represent means for species $i$ and that their
among-individual distributions are symmetrical with additive genetic
variance $\sigma_{A}^{2}$. Assuming also that $\sigma_{A}^{2}$ is
relatively small \citep{Abrams1993b, Abrams2001a, Iwasa1991a},
the evolution of traits are given by the equations



\begin{equation} \label{eq:invest-change}
\begin{split}
    p_{i,t+1} &= p_{i,t} + \left( \frac{1}{F_i}
        \frac{\partial F_i}{\partial p_{i,t}} \right) \sigma^2_A \\
        &= p_{i,t} + 2 \, \sigma_A^2 \left[
            \alpha_0 \, \text{e}^{-p_{i,t}^2 - x_{i,t}^2} \, p_{i,t} \,
            \sum_{j \ne i}^{n}{ N_{j,t} \, \text{e}^{ 
                -d_p \, p_{j,t}^2 + d_x \, x_{j,t}^2 } }
            - f \left( p_{i,t} + \eta \, x_{i,t} \right) 
        \right]\\[1ex]
    x_{i,t+1} &= x_{i,t} + \left( \frac{1}{F_i}
        \frac{\partial F_i}{\partial x_{i,t}} \right) \sigma^2_A \\
         &= x_{i,t} + 2 \, \sigma_A^2 \left[
            \alpha_0 \, \text{e}^{-p_{i,t}^2 - x_{i,t}^2} \, x_{i,t} \,
            \sum_{j \ne i}^{n}{ N_{j,t} \, \text{e}^{ 
                -d_p \, p_{j,t}^2 + d_x \, x_{j,t}^2 } }
            - f \left( x_{i,t} + \eta \, p_{i,t} \right) 
        \right]
    \textrm{.}
\end{split}
\end{equation}



\subsection*{Stability}

The ecological and evolutionary stationary states produced by our model
can be convergence stable and/or ESS stable \citep{Geritz1998}.
A stationary (i.e., time-invariant) state is convergence stable if a
nearby state evolves towards the stationary state. A stationary state is
ESS-stable if, when occupied by the system, any mutant away from the
state has lower fitness than those phenotypes at the stationary state.
Convergence-stability does not imply ESS-stability, and vice versa
\citep{Geritz1998}. Because we iterated the model until
abundance and trait values reached a stationary state, the stationary
state is convergence stable. To check for ESS convergence simultaneously
for abundance and trait variables, we computed the Jacobian matrices for
the abundances and investments of each species (eq. \ref{eq:jacobian}). 
The dominant eigenvalue of this matrix ($\lambda$) corresponds to a stationary
state that is stable when $\left\lVert\lambda\right\lVert < 1$, 
neutrally stable when $\left\lVert\lambda\right\lVert = 1$, and 
unstable when $\left\lVert\lambda\right\lVert > 1$. Analytical
expressions for the derivatives are shown in Appendix B.

\subsection*{Non-heritable variation}

In our model, changes in trait investments are driven by selection on
additive genetic variance. However, in natural populations many factors other
than additive genetic variance are likely to influence trait values that
underpin competition, most notably plasticity \citep{Miner2005}.
To explore the effects of non-heritable variation on trait coevolution, we
modeled the expected values of the phenotypes ${\ddot{p}}_{i}$ and
${\ddot{x}}_{i}$ as the products of the latent additive-genetic
components ($p_{i}$ and $x_{i}$) and log-normal perturbations, such
that
${\ddot{p}}_{i,t + 1} = p_{i,t + 1}\mspace{6mu}\text{e}^{\varepsilon_{p,t + 1}}$
and
${\ddot{x}}_{i,t + 1} = x_{i,t + 1}\mspace{6mu}\text{e}^{\varepsilon_{x,t + 1}}$,
where $\varepsilon_{p,t + 1}$ and $\varepsilon_{x,t + 1}$ are
generated from normal distributions with means of 0 and variances
$\sigma_{p}^{2}$ and $\sigma_{x}^{2}$, respectively. Because
phenotypes affect both competition and associated costs, they are used
to calculate all species' fitness values and the effects of investments
on competition. Genotypes change through time according to the effect of
phenotypes on per-capita growth:


\begin{equation} \label{eq:invest-change-non-heritable}
\begin{split}
    p_{i,t+1} &= p_{i,t} + \left( \frac{1}{F_i}
            \frac{\partial F_i}{\partial \ddot{p}_{i,t}}
            \frac{\partial \ddot{p}_{i,t}}{\partial {p}_{i,t}}
        \right) \sigma^2_A \, \text{e}^{\varepsilon_{p,t+1}} \\
    x_{i,t+1} &= x_{i,t} + \left( \frac{1}{F_i}
            \frac{\partial F_i}{\partial \ddot{x}_{i,t}} 
            \frac{\partial \ddot{x}_{i,t}}{\partial {x}_{i,t}} 
        \right) \sigma^2_A \, \text{e}^{\varepsilon_{x,t+1}}
    \textrm{.}
\end{split}
\end{equation}

\noindent where $\frac{\partial{\ddot{p}}_{i,t}}{\partial p_{i,t}}$ and
$\frac{\partial{\ddot{x}}_{i,t}}{\partial x_{i,t}}$ equal 1. We
modeled non-heritable variation as simple perturbations because we were
interested in the role of this variation per se---and not, for example,
adaptive plasticity---and because non-heritable variation is often not
adaptive \citep{Ghalambor2007, Miner2005}. The form of
non-heritable variation we modeled could be produced by plasticity induced
by stochastic environmental change. For example, temperature can change
both nutritional demands and foraging behavior, so the costs and
benefits of partitioning and conflict traits will likely depend on
temperature.

\subsection*{Simulations}

We used simulated communities with 2 to 4 species to explore the behavior 
of the model. We do not present more species-rich communities because under 
the conditions we simulated, 2--4 species were sufficient to illustrate all 
of the qualitative outcomes of the model: for simulations with more species,
species clustered into groups that were represented in our simulations of 
2--4 species. To highlight different qualitative outcomes, we assumed
symmetry among species by setting parameters $r_{0}$, $\alpha_{0}$,
and $f$ to the same values for all species; using different values for
different species changes the quantitative but not qualitative outcomes.
All species started with an abundance of 1 and were considered extinct
if their abundance fell below 1. We ran simulations long enough for
communities to reach stationarity ($\ge$~10,000 time steps). 
Our R package
\texttt{sauron}
(\url{https://anonymous.4open.science/r/sauron-F8CC}) 
contains the code for the simulations that uses a combination of R
\citep{RCoreTeam2022} and C++, the latter via the Rcpp and
RcppArmadillo packages \citep{Eddelbuettel2014a, Eddelbuettel2013a,
Sanderson2016}.


