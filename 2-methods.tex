
\section*{Material and methods}


\subsection*{Model overview}

We used a simple model of competition to evaluate how evolving investment in 
competition affects coexistence among species.
Each of $n$ species has $q$ competition axes.
Competition axes are combinations of multiple traits that together affect 
competition similarly.
Investing in an axis has a cost to the population growth rate and 
a benefit via reduced interspecific and intraspecific competition.
The overall competition experienced by each species is a product of 
the axes in all species in the community.
This is because all species have at least two axes, 
one conflicting, and one ameliorative.
For a conflicting competition axis, investment by one species 
strengthens competition for all other species.
For an ameliorative axis, all species' competition is reduced when
one species invests in it.


All species are symmetrical in that when species share the same axis 
values, they will always have the same per-capita effect on each other.
The benefits and tradeoffs are the same for all axes.
In both the costs and benefits, axis effects are concave functions that,
combined, ensure that one fitness peak exists for each axis.
Axes can also have non-additive trade-offs that either increase or decrease
the cost associated with increasing multiple axes compared to increasing
just one \citep{Northfield2021}.

% Here, I think it makes sense to define what we mean by axes:
We could expect conflicting evolution to occur when contest competition
leads to arms races among competitors
\citep{Abrams1994}.
Competitor evolution might be ameliorative when they evolve
dissimilar resource-usage traits to reduce competition \citep{Roughgarden1976}.


We used a discrete-time, modified Lotka--Volterra competition model similar to
that by \citet{Northfield2013a}.
In it, species $i$ has a length-$q$ vector of axes ($\mathbf{v}_i$), and
its per-capita growth---equivalent to fitness---is

\begin{equation} \label{eq:fitness}
    F_{i} = \exp \left\{ r_i(\mathbf{v}_i) - 
        \alpha_{ii}(\mathbf{v}_i) \, N_i - \sum_{j \ne i}^{n}{
            \alpha_{ij}(\mathbf{v}_i, \mathbf{v}_j) \, N_j}  
    \right\}\textrm{,}
\end{equation}

\noindent where $N_i$ is the population density of species $i$.
The parameter $r(\mathbf{v}_i)$ describes how axes affect
the growth rate:

\begin{equation} \label{eq:growth-rate}
\begin{split}
    r(\mathbf{v}_i) &= r_0 - f \, \mathbf{v}_i^{\textrm{T}} \, \mathbf{C} ~ \mathbf{v}_{i} \\
    \mathbf{C} &= \begin{pmatrix}
        1         & \ldots & \eta_{1q} \\
        \vdots    & \ddots & \vdots \\
        \eta_{q1} & \ldots & 1      \\
        \end{pmatrix}
    \textrm{,}
\end{split}
\end{equation}

\noindent where $r_0$ is the baseline growth rate,
$f$ is the cost of increasing axes on the growth rate, and
$\eta_{k,l}$ is the non-additive trade-off of increasing both the
$k$\textsuperscript{th} and $l$\textsuperscript{th} axes.
When $\eta > 0$, increasing multiple axes incurs an extra cost.
Non-additive trade-offs are symmetrical (i.e., $\eta_{k,l} = \eta_{l,k}$ for all
$l$ and $k$), and all values on the diagonal of $\mathbf{C}$ are 1.


The terms $\alpha_{ii}(\mathbf{v}_i)$ and
$\alpha_{ij}(\mathbf{v}_i, \mathbf{v}_j)$
in equation \ref{eq:fitness} represent how axes influence the effects
of intraspecific and interspecific competition, respectively.
Competitive effects are given by

\begin{equation} \label{eq:competition}
\begin{split}
    \alpha_{ii}(\mathbf{v}_i) &= \alpha_0 ~\exp \left\{
        - \mathbf{v}_i^{\textrm{T}}
        \mathbf{v}_i \right\} \\
    \alpha_{ij}(\mathbf{v}_i, \mathbf{v}_j) &= \alpha_0 ~\exp \left\{
        - \mathbf{v}_i^{\textrm{T}} \mathbf{v}_i -
        \mathbf{v}_j^{\textrm{T}} \mathbf{D} \mathbf{v}_j \right\} \\
    \mathbf{D} &= \begin{pmatrix}
        d_1     & \ldots    & 0 \\
        \vdots  & \ddots    & \vdots \\
        0       & \ldots    & d_q
        \end{pmatrix}
	\textrm{,}
\end{split}
\end{equation}



\noindent where $\alpha_0$ is the base density dependence.
Matrix $\mathbf{D}$ contains parameters that determine how evolution of axes
in one species affects competition experienced by others:
When $d_k < 0$, investment in by species $i$ in axis $k$ decrease the
effect of competition on species $i$, but increase it in all others
(i.e., axis $k$ is conflicting).
Alternatively, when $d_k > 0$, the same investment decreases the effect of
competition on all species including $i$ (i.e., axis $k$ is ameliorative)
\citep{Northfield2013a}.


The relationships between axis values and other components of fitness
(growth rates and effects of competition) are of the form
$X \propto v^2$ for parameter $X$, so $v = z$ is equivalent to $v = -z$.
To avoid alternative outcomes due to artifacts of this relationship,
axes are not allowed to be $< 0$.
We did this by passing the equation for $\mathbf{v}_{t+1}$ through a
ramp function.
We used a ramp function instead of absolute values
because the latter causes fluctuations
in the axis values when they approach zero (they ``bounce off''
the zero bound) that persist for a very long time;
this caused the simulations to take a prohibitively long time to reach
equilibrium.
A more important disadvantage is that $d \lvert x \rvert / dx$ is
undefined when $x = 0$.
This implementation and its consequences on resulting derivatives are in
Appendix A.




% \subsection*{Adaptive dynamics}
%
% We started simulations with a single competitive species with axis values set to zero.
% We tracked species population densities through time using equation \ref{eq:fitness} and
% considered a species extinct if its density fell below $10^{-4}$.
% Species produced daughter species with a probability of 0.01 per species per time step.
% We generated daughter-species axis values from normal distributions with means of the
% mother axis values and standard deviations of $\sigma_{d}$.


\subsection*{Quantitative genetics}

We used a quantitative genetics framework for axis evolution.
We assumed that all axes in $\mathbf{v}_i$ represent means for species $i$
and that their among-individual distributions are symmetrical with additive
genetic variance $\sigma^2_i$.
Assuming also that $\sigma^2_i$ is relatively small
\citep{Iwasa1991a,Abrams2001a,Abrams1993b} and that

\cite{DeVries2019}


, axes at time $t+1$ are

\begin{equation} \label{eq:axis-change}
    \mathbf{v}_{i,t+1} = \mathbf{v}_{i,t} + \left( \frac{1}{F_i}
        \frac{\partial F_i}{\partial \mathbf{v}_{i,t}} \right) \sigma^2_i
    \textrm{.}
\end{equation}

To determine the stability of ending points (axis values and abundances of
surviving competitor(s)), we computed the $n (q+1) \times n (q+1)$ Jacobian matrices
of first derivatives for the axes and abundances of each species (Equation \ref{eq:jacobian}).
We then computed the primary eigenvalue of this matrix ($\lambda$).
We considered a state stable when $\lambda < 1$,
neutrally stable when $\lambda = 1$,
and unstable when $\lambda > 1$.

Full analytical solutions to matrix derivatives (for axis change and
Jacobian matrices) are found in appendix A.
We also analyzed equilibrium solutions for the 2 axis case.
This is found in Appendix B.


\subsection*{Stochasticity}

We added stochasticity to population dynamics by simply adding 
a log-normal error term:

\begin{equation} \label{eq:N-stochasticity}
\begin{split}
    N_{i,t+1} &= N_{i,t} \, F_{i,t} \; \text{e}^{\varepsilon_N} \\
    \varepsilon_N &\sim \text{N}(0, \, \sigma^2_N)
    \text{.}
\end{split}
\end{equation}


Stochasticity for axis evolution takes the form of non-adaptive
phenotypic plasticity,
where phenotypes ($\mathbf{\ddot{v}}$) are the product of the
genotypes ($\mathbf{v}$) and a log-normal error term:

\begin{equation} \label{eq:V-stochasticity}
\begin{split}
    \mathbf{\ddot{v}}_{i,t+1} &= \mathbf{v}_{i,t+1} \; \text{e}^{\varepsilon_V} \\
    \varepsilon_V &\sim \text{N}(0, \, \sigma^2_V)
    % \varepsilon_V &\sim \text{N}(\mu_V, \, \sigma^2_V)
    \text{.}
\end{split}
\end{equation}

% \noindent To keep stochasticity from affecting the direction of axis evolution, 
% we set $\mu_V$ to $-\sigma^2_V / 2$, so that the mean of $\text{e}^{\varepsilon_V}$ was 1.

Because the phenotypes interact with the environment, they are used
to calculate all species' fitness values.
Genotypes change through time based on the previous time point's 
genotypes and how the phenotypes affect fitness:

\begin{equation} \label{eq:axis-change-stochastic}
    \mathbf{v}_{i,t+1} = \mathbf{v}_{i,t} + \left( \frac{1}{F_i}
        \frac{\partial F_i}{\partial \mathbf{\ddot{v}}_{i,t}} \right) \sigma^2_i
    \textrm{.}
\end{equation}





\subsection*{Simulations}


We started simulations with one species, and each of $n-1$ new species
was added every 500 generations.
Species started with an abundance of 1.
We continued simulations for another 20,000 generations after all
species were added.
We tracked densities through time and considered a species extinct if its 
density fell below $10^{-4}$.

% Species starting axis values were generated from truncated normal 
% distributions with standard deviations of 1 and lower bound of zero.
% The means varied by the type of simulation.




% Each species had starting axis values generated from a truncated normal distribution 
% with a mean equal to the equilibrium solution for that axis,
% a standard deviation of 1, and a lower bound of zero.

% Species started with an abundance of 1.


% When generating multiple $\eta$ magnitudes for the different tradeoffs,
% we did so using a uniform distribution from 0.1 to 0.4.
% We intentionally avoided the use of simple $\eta$ values because
% when the sum of one or more $\eta$ equals another, this simplifies
% to additivity even when those $\eta \ne 0$.
% This only applies in the case of $q > 2$, since there is only one
% $\eta$ value when $q = 2$.
% When $q = 2$, we chose $\eta = 0.6$ because it was sufficiently large
% to show how its value changes the pattern of outcomes.

% For simulations that aim to find points in axis space associated with
% species surviving competitive environments, we set $d = 1$ 
% (ameliorative evolution) so that many species
% survived and surveyed the axis space more effectively.
% This has the added benefit of being a symmetric case where
% all species in the community benefit equally when one
% species evolves an increased axis.


% We first simulated 2-axis communities to survey many of the
% basic properties of the model.
% Next, we simulated 3- and 4-axis communities.
% With 3 axes, there is more than one tradeoff, so we could
% explore combinations of different types of additivity.
% With 4 axes, there are more tradeoffs than axes, which we
% thought might cause different possibilities in the model.



\subsection*{Code}

We simulated models using a combination of R \citep{RCoreTeam2020} and
C++ via the Rcpp and RcppArmadillo packages
\citep{Eddelbuettel2014a,Eddelbuettel2013a,Sanderson2016}.
We double-checked our derivations by simulating 100 datasets
(4 species with 3 axes each) and computing derivatives using the Theano Python
library's automatic differentiation \citep{TheanoTeam2016a}.
All code can be found on GitHub
(\url{https://github.com/lucasnell/sauron}).

