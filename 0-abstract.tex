\section*{Abstract}

Coevolution of competitors can lead to niche partitioning promoting
coexistence or to heightened conflicts promoting competitive exclusion. 
If both are possible, when should coevolution favor coexistence versus
exclusion? We investigated this question with a general eco-evolutionary 
model in which species can reduce the interspecific competition they 
experience through evolutionary investments in two types of competitive traits:
partitioning traits that promote coexistence and conflict traits that promote
exclusion. We found that communities were generally mixed, consisting of species
investing in both trait types or mixtures of species specializing in one type.
For each species, its competitors’ abundances and investments determined its
experienced competition, and stronger competition begot greater competitive
trait investment. Species investing in conflict traits strengthened competition
for other species both directly and indirectly, whereas partitioning traits only
weakened competition via direct effects. Conflict traits were therefore the
stronger driver of community-wide investments in all traits. However, species
investing most in conflict traits experienced less competition, so they
ultimately evolved least investment, making them most likely to be excluded by
the next invader. Thus, coevolution may provide an open door for species that
play nice and a revolving door of exclusion for those that do not. 
