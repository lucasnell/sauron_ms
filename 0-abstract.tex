\section*{Abstract}


%% \textbf{Should be $\le$ 150 words for \emph{Ecology Letters}}

\textbf{This is 199 words right now. I'm shooting for $\le$ 150.}


Theory has yielded many mechanisms that can affect coexistence of 
coevolving, competing species. 
Yet these models typically focus on one or few mechanisms. 
We analyze a general model of competitive communities where species can evolve 
to reduce interspecific competition from competitors by investing 
in ``competition axes'' that also affect competition experienced by 
others in the community. 
It also includes stochastic, non-adaptive plasticity to assess how 
environmental effects influence coexistence. 
For a given number of species, we find that the suite of alternative 
stable communities is largely determined 
by tradeoffs for investing in multiple axes (compared to just one) 
and by how axis investment affects other competitors. 
Tradeoffs, axis effects, and the investments made by residents 
together determine community invasibility, 
although axis effects can switch direction with increasing 
strength due to changes in community stability. 
Species investments also have a feedback where increasing numbers 
of species investing in axes that increase competition for others 
causes each to invest more. 
Lastly, changes in stochasticity acted much like changes in 
tradeoffs, moving attractors through axis space. 
Our results help us understand how the general properties underlying 
competition affect coexistence among coevolving species. 
These results span many of the familiar mechanisms affecting coexistence.



% This is an overly-brief version of what you should write.
% (It's from your SSF application.)

I use a general theoretical model to assess how coevolution should  
affect coexistence among competing species. In contrast to previous models,  
I allowed  species to make evolutionary “investments” in two types of  
traits, those that stabilize coexistence and those that destabilize it.  
I find that coevolution should often result in mixed communities,  
where some species invest in stabilizing (e.g., niche partitioning) and  
others in destabilizing (e.g., aggressive resource defense) traits.  
Investments by one species affect the fitness landscape and resulting 
investments by other species, with potential positive feedback loops.  
This adds to the evidence that evolution can often inhibit coexistence,  
and further complicates the question of how so many species coexist in 
nature. 
