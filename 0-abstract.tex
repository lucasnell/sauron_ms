\section*{Abstract}


\textbf{Should be $\le$ 150 words for \emph{Ecology Letters}; it's 181 now}


The long history of research on coevolving, competitive communities has resulted in 
many mechanisms that affect coexistence among species.
Here, we provide general insights into this process
using a simple model of competitive communities where species can evolve to reduce 
competition by investing in competition axes.
Investment in the ameliorative axis also weakens competition for other species, 
while in the conflicting axis, it strengthens it.
In deterministic simulations, 
tradeoffs for investing in multiple axes (compared to just one) and 
species starting axis values determined the axis states to which species evolved.
The relative strength of axes dictated whether axis states were permissive
to coexistence among species.
Adding stochasticity typically caused a mild decrease in the chances for
coexistence, via negative transient effects on invaders.
However, when tradeoffs were weak,
stochasticity in axis evolution could have strong effects on
coexistence due to the effects of adding variance to the non-linear
equations for axis evolution.
Our results help us understand how the general properties underlying
competition affect coexistence among coevolving species.
These results span many of the well known mechanisms that affect coexistence.




