\section*{Abstract}


%% \textbf{Should be $\le$ 200 words for \emph{The American Naturalist}}

Coevolution of competitors can lead to niche partitioning that 
promotes coexistence or to traits that cause conflict between competitors, 
which would promote exclusion.
When should coevolution result in traits that contribute to niche 
partitioning versus competitive exclusion? 
We investigated this question with a general eco-evolutionary model in 
which species can reduce their experienced effects of interspecific 
competition by making evolutionary investments in two suites of traits: 
partitioning traits that promote coexistence and 
conflicting traits that promote exclusion.
The cost cost for simultaneously investing in both suites of traits
determines whether species will invest in one or both suites.
The total investment by each species is unsurprisingly dictated by 
the cost/benefit ratio of investment. Benefits are determined by both 
the number of competitors and the investments each has made, so 
alternative community states greatly affect investments by new species. 
In communities dominated by investment in partitioning traits, 
competition is weakened and per-species investment decreases, 
which causes
(1)~a ceiling for the community-wide investment in partitioning traits and
(2)~resistance to a single conflicting species added to a community of
partitioning investors.
However, just two conflicting investors can cause an arms race
that increases investment for all in the community if their arrivals in
the community are close in time.
Large time gaps between conflicting invaders causes serial exclusion of 
invaders invested in conflicting traits and easier invasion for those
invested in partitioning traits.
This could result in the accumulation of species that evolve niche partitioning
and filtering out of those whose evolution would promote exclusion.
