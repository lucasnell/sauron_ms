\section*{Abstract}


%% \textbf{Should be $\le$ 150 words for \emph{Ecology Letters}; it's XXX now}


The long history of research on coevolving, competitive communities has resulted in 
many mechanisms that affect coexistence among species.
Here, we provide general insights into this process
using a simple model of competitive communities where species can evolve to reduce 
interspecific competition by investing in ``competition axes.''
These axes are combinations of traits that directly affect competition
for the species possessing these traits and for others in the community.


Investment in the ameliorative axis weakens competition for other species, 
while in the conflicting axis, it strengthens it.
Tradeoffs for investing in multiple axes (compared to just one),
species' starting axis values, the number of species in the community,
and the investments each species made
together determined the axis states to which species evolved.
If, for instance, some competitors invested in a strong
ameliorative axis, then selection would favor no investment
at all by species whose only evolutionary option is to invest in
a conflicting axis.
Adding stochasticity typically caused a mild decrease in the chances for
coexistence, via negative transient effects on invaders.
However, when tradeoffs were additive or weakly non-additive,
stochasticity in axis evolution could have strong effects on
coexistence due to the effects of adding variance to the non-linear
equations for axis evolution.
Our results help us understand how the general properties underlying
competition affect coexistence among coevolving species.
These results span many of the well known mechanisms that affect coexistence.




