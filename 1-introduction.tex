
\section*{Introduction}


Competitive communities are formed through both ecological and evolutionary processes.
The ecologically relevant traits available to competitors are a result of 
evolutionary history and of ongoing evolution.
How the use of these traits affects others in the community, and
how these effects feed back to affect the focal species,
are ecological processes.
Given our somewhat recent understanding that ecological and evolutionary processes
can occur on the same timescales, theoretical models increasingly
include both components, and they are providing new insights into how species coexist.

% Stabilizing mechanisms:
Many mechanisms can allow for coexistence among competitors.
Often in theoretical studies of competitive coexistence, 
an emphasis is placed on differences among competitors
along a resource gradient.
Greater differences among competitors in their use along the gradient reduces
the effect of competition among them, which provides a density-dependent 
stabilizing effect and increases the ability for them to coexist.


% Destabilizing mechanisms:
However, these coexistence mechanisms do not exist in an ecological vacuum.
Even when coexistence mechanisms exist between competitors, other
ecological interactions between them may undermine their coexistence.
Even if two birds evolve to reduce spatial overlap in foraging areas,
selection may also favor increased aggression towards heterospecifics
when they do interact.
Similarly, not all species in a community are likely to evolve
traits contributing to mechanisms that stabilize coexistence.
If two plants evolve to assimilate nutrients at different ratios,
but a third plant evolves allelopathy, 
how does the latter affect coexistence among all three species?
When these complexities might arise and how they affect coexistence is 
not obvious.


Here, we use a simple model to describe what happens when competitors'
coevolution can strengthen both stabilizing and destabilizing mechanisms.
In contrast to many other models, we do not impose a particular mechanism
of coexistence (e.g., niche breadth) in our model.
In our model, species can evolve to reduce the effects they experience from
competition by investing in ``competition axes.''
These axes directly affect their per-capita growth rates by reducing the 
density-dependent effects they experience from other species.
Axes take one of two forms: 
Investment by species $i$ either reduces the effects of competition on all other species (an ``ameliorative axis'') or 
it increases it (a ``conflicting axis'').
Investment also has a cost, and this cost can be either more or less
costly when a species invests in two or more axes.
Using this general framework, we explore when coexistence is most likely,
and we compare these results to some of the other coexistence mechanisms
that are often discussed.












































% Alternative community states can be described by both the number of species present and by the
% ecologically relevant traits that those species possess. How communities arrive at these states
% is a longstanding question in community ecology
% \citep{Drake:1991bv,Weiher:1999tf,Gleason:1927cj,Clements:1936hw}.
% How community processes shape trait evolution and species filtering is an equally enduring
% question in evolutionary ecology
% \citep{Darwin:1859to,Loeuille:2018cx,Pontarp:2018hv,MacArthur:1964uv,Schluter:2000jz,Muschick:2012ha}.
% Despite the length of time spent on these topics, few generalizations exist \citep{Lawton:1999fj}, at
% least partly because of context-dependent mechanisms \citep{Drake:1991bv} and the complex effects of
% history \citep{Drake:1991bv,Chase:2003ko,Weiher:1999tf}.
%
% Competition has been a particularly well-studied process for shaping communities and trait evolution
% \citep{Simpson:1953wr,Volterra:1928fy,Macarthur:1964kv,Hardin:1960ep,Roughgarden:1976eh,Rosenzweig:1978bj,
% Armstrong:1980id,Hutchinson:1959tq,BrownJr:1956wi,Day:2004db}.
% In many theoretical models, competition strength between two species is inversely proportional to
% the difference in their ecologically relevant trait values
% (\citealp{Abrams:1983jz,Macarthur:1967jf,Volterra:1928fy,Macarthur:1964kv,Rosenzweig:1978bj};
% reviewed in \citealp{Taper:1992kz,Taper:1985ub,Abrams:1986tx,Dayan:2005ub}).
% Models can include this relationship either explicitly \citep[e.g.,][]{Burger:2006tq,Roughgarden:1976eh,Zu:2008uw}
% or implicitly, where trait values represent the ability to extract different resources
% \citep[e.g.,][]{Macarthur:1964kv,Ackermann:2004bb}. In either case, this process can generate or maintain
% trait diversity in a range of forms, including alternative stable states and limit cycles
% \citep{Gilpin:1975gz,Burger:2006tq}.
% In other models, traits change how species perform in competitive contests.
% This can often lead to arms races, where traits cycle or continually increase
% \citep{MaynardSmith:1986tw,Parker:1983io}.
% When \citet{Abrams:1994th} explicitly included population dynamics into a similar model, more complex
% patterns emerged, such as dimorphism and alternative stable states.
%



% Ecological limits to evolutionary rescue


