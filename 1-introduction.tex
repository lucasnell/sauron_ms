
\section*{Introduction}

Interactions among competing species influence their ability to coexist.
Those that weaken interspecific competition in relation to
intraspecific competition should help stabilize coexistence,
while those doing the opposite should help destabilize it 
\citep{Chesson2000}.
Mechanisms that should cause a relative decrease in species'
per-capita effects on their competitors should help stabilize coexistence.
These often involve some form of niche partitioning, 
where species differ in the factors that affect their population growth rate.
Classic examples include species using different resources
\citep{Macarthur1967a,Roughgarden1976,Abrams2009},
but other differences among species, such as their 
effects from natural enemies
\citep{Grover1998,Vandermeer1998,Ehrlich2017a,Abrams2002}
or responses to environmental change
\citep{Chesson1997,Armstrong1976,Loreau1992,Chesson1994,Kremer2017,Pacala1994}
may also have similar effects.
Alternatively, mechanisms that can increase species' per-capita effects on 
their competitors should destabilize coexistence.
These typically occur when species possess traits that cause conflict
between their interests and those of their competitors.
One example of a conflicting trait is plant height, since taller plants 
more successfully access light but shade their shorter competitors.
This can lead to an arms race among plants for increased height, 
within the constraints imposed by the physiological cost of being tall
\citep{Falster2003}.
Other examples of conflicting traits include
seed size \citep{Geritz1999,Fagerstrdm2016},
aggressive behavior \citep{Brown1971},
and body size \citep{Kisdi2001}.
Tradeoffs and frequency dependence can allow for coexistence despite
among-species variation in these traits
\citep{Kisdi2001,Falster2003}, but in isolation, the effects these traits have
on other species act to limit stable coexistence.


The traits underlying both conflicting and partitioning mechanisms 
are thus important forces mediating competitive interactions that 
affect coexistence.
Under what conditions should they evolve?
Despite the long history of research on competition's effects on traits
\citep[e.g., ][]{Schreiber2011,Taper1992,Pastore2021,Abrams1983a,Roughgarden1976,Vasseur2011a}, 
we know much less about how traits that either stabilize or destabilize
coexistence result from coevolution among competitors in communities of 
mixed trait types.
For example, what happens to resource partitioning among some species in a
community when others have evolved to instead aggressively defend resources?
Alternatively, how are arms races among competitors affected by a subset of 
the community that vary in their responses to environmental change?
Feedbacks between eco-evolutionary elements of communities combined with 
the fact that the effects of species interactions can depend on context
\citep{Saavedra2017, Song2020}
make these questions non-trivial.
Moreover, both types of traits are likely ubiquitous in competitive communities.
In tropical forest tree communities, competitors have evolved to better capture
light (and thereby shade competitors), but many mechanisms, such as
species-specific responses to herbivores, also play roles in maintaining 
the rich species diversity seen here \citep{Wright2002}.







% mention these as limitations? (maybe also fluctuation-dependent mechanisms?)
% storage effect \citep{Warner1985,Abrams2013}
% frequency-dependent predation \citep{Gendron1987}

Here, we use a simple model of coevolving competitors to evaluate how 
partitioning and conflicting traits evolve in a community.
Competitors can evolve to reduce the interspecific competition they experience
by investing in two suites of competitive traits, where suites are 
distinguished by their effects on other competitors
(Figure \ref{fig:model-description}).
In the first suite (``conflicting traits''), investment by species $i$
strengthens the competitive effect that species $i$ has on all
other competitors.
% Investment in this suite would promote competitive exclusion and 
% might occur when contest competition leads to arms races among competitors
% \citep{Falster2003}.
In the second suite (``partitioning traits''), all competitors' competition
from species $i$ is reduced when it invests in this suite.
% This would promote competitive coexistence, and might occur when species evolve
% to exploit different resources from other competitors
% \citep{Roughgarden1976}.
% These across-species effects of investing mean that the overall 
% competition experienced by any species is affected by 
% the investments made by all community members.
Investment also incurs a cost to the per-capita growth rate, and 
these costs can be non-additive for simultaneous investments in both suites.
For sub-additive costs, it is less costly to invest in both suites, and
for super-additive, it is more costly.
As an illustrative example, we can return to plant height, where we
consider a forest with trees of varying heights.
If the shading of taller trees increases the competition experienced by 
the shorter ones, the cost of maintaining a tall stem would represent the 
tall trees' investments in conflicting traits.
To decrease the overall competition they experience from tall trees,
short trees might evolve to use different below-ground resources.
If this change in resource use has a fitness cost and if the tall trees also
experience reduced competition from the short trees as a result, then 
the short trees are investing in a partitioning suite of traits.
If having both a tall canopy and different resource use is more costly
than the summed cost of both separately, then costs are super-additive.



Each suite of traits is modeled as one unit and only in terms of its 
effects throughout the community, so
investments do not imply any specific changes in values of the
underlying traits.
Within the constrains of needing to act together, what traits can
constitute a suite are quite flexible:
In the example above, the partitioning suite of traits that allows for 
resource partitioning could be the physiological traits associated with 
relative uptake rates of various nutrients, or it could be a derived trait 
such as average resource use along a gradient.
This abstraction also means that (1) investments do not define species
(i.e., species with the same investments are not considered the same species)
as they often do in eco-evolutionary models \citep[e.g., ][]{Northfield2021}
and
(2) similarity in investment has no effect on the strength with which 
species compete.
Additionally, all species are symmetrical in that when species make the
same investments, they will always have the same per-capita effect on
each other.
Under these assumptions, we seek general insights into how conflicting 
and partitioning traits result from coevolution among competitors.









% \citep{Doebeli2010a}


% eco-evo affecting competition: \citep{Pantel2015}

% Evolution of ecologically relevant traits---and 
% associated species interactions---adds to this complexity,
% since it can have strong effects on coexistence
% \citep{Wagner2017a,TerHorst2018}.
% Community dynamics are affected by evolutionary change
% for initial assembly \citep{Vellend2010},
% subsequent invasions \citep{Faillace2016},
% and long-term equilibria, both
% stationary \citep{Lankau2011,Barabas2016}
% and cyclical \citep{Fussmann2013,Vasseur2011a,Kremer2017}.
% Evolution can contribute to coexistence by
% improving invasion success \citep{Faillace2016}
% or buffering against environmental perturbations 
% \citep{Barabas2016,Fussmann2013,Osmond2013,Bell2017}.
% More complex mechanisms of evolution mediating coexistence, 
% such as synergistic pleiotropy \citep{Schreiber2018a}
% or ``neighbor-dependent selection'' \citep{Vasseur2011a},
% have also been demonstrated.
% Alternatively, some evidence suggests that evolution can 
% reduce species persistence \citep{Ferriere2013},
% increase exclusion of invaders by resident species \citep{Faillace2016},
% and destabilize long-term coexistence \citep{Shoresh2008}.
% Recently, \citet{Pastore2021} found that evolution often caused
% exclusion in a 2-species, consumer--resource model, 
% largely due to evolutionary changes in competitive differences.


