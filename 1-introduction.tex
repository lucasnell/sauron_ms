
\section*{Introduction}

Interactions among competing species can either promote or inhibit
coexistence. Those that weaken interspecific competition relative to
intraspecific competition should promote coexistence, while those that
do the opposite should inhibit it \citep{Chesson2000}. Traits
that promote coexistence often entail some form of niche partitioning,
where species differ in the factors that affect their population growth
rate (i.e., have different niches). Classic examples
include species using different resources
\citep{Abrams2009, Macarthur1967a, Roughgarden1976}. However,
any trait affecting a population's growth rate can serve as the basis
for niche partitioning, including interactions with natural enemies
\citep{Abrams2002, Ehrlich2017, Grover1998, Vandermeer1998} and
responses to environmental change \citep{Armstrong1976, Chesson1994,
Chesson1997, Kremer2017, Loreau1992, Pacala1994}. Alternatively,
traits that increase species' per capita effects on their competitors
should inhibit coexistence. These typically occur as traits that cause 
conflict between a species and its competitors. For example, taller plants 
can more successfully access light and shade their shorter competitors, 
leading to an arms race among plants for increased height within the 
constraints imposed by the physiological cost of being tall
\citep{Falster2003}. Other examples of conflict traits include
seed size \citep{Fagerstrdm2016, Geritz1999}, agonistic behavior
\citep{Brown1971}, and body size \citep{Kisdi2001}.
While investment in conflict traits does not necessarily lead to
competitive exclusion \citep{Falster2003, Kisdi2001},
competitive exclusion can result when investment is sufficiently high.

Under what conditions should coevolution of competitive traits lead to
coexistence versus exclusion? This depends on the extent to which species 
invest in partitioning traits (those that reduce per capita interspecific
competition on both itself and its competitors) versus conflict traits
(those that reduce per capita interspecific competition on itself while
increasing competition on its competitors). Most theoretical studies focus 
on partitioning traits taking a specific form, such as traits affecting 
resource use
\citep[e.g.,][]{Roughgarden1976, Shoresh2008} or apparent
competition \citep[e.g.,][]{Abrams1998, Abrams2002, Schreiber2011}.
Although some studies have explored the effect of coevolution on 
competitors’ traits using more general forms of competitive 
interactions \citep[e.g.,][]{Abrams2013, Pastore2021, Vasseur2011a}, these
models consider only the evolution of partitioning traits. Yet, both
partitioning and conflict traits are likely ubiquitous in competitive
communities. In tropical forests, for example, trees have evolved to
better capture light (and thereby shade competitors), but many
mechanisms, such as species-specific responses to herbivores, lead to
partitioning traits \citep{Wright2002}. Additionally, although coevolution of
competitors is most often thought to promote niche partitioning 
\citep{Pfennig2012}, experimental evidence has shown that either niche
partitioning \citep{Schluter1994, Zuppinger-Dingley2014} or
exclusion-promoting traits \citep{Germain2020, Hart2019, Miller2014, Zhao2016}
can evolve in response to competition. 
Therefore, understanding how coevolution might lead to coexistence or
exclusion of competing species requires models that include both
partitioning and conflict traits.

Here, we use a general model of coevolving competitors to evaluate how
partitioning and conflict traits evolve, and how coevolution affects the
structure of communities. In the model, each species can reduce the
interspecific competition it experiences by investing in partitioning
and/or conflict traits, and the level of investment evolves through
selection. Investment in partitioning traits by species $i$ reduces
the competitive effect that it has on all other competitors.
Investment in conflict traits by species $i$ increases the competitive
effect that species $i$ has on all other competitors. Investment
incurs a cost to a species' per capita growth rate that does not
directly involve the competitive effects. We take a quantitative
genetics approach to model selection on trait investment, and the model
allows both additive-genetic and non-heritable variation. For
additive-genetic trait variation, the costs for simultaneous investments
in both partitioning and conflict traits can be non-additive. In the
sub-additive case, the overall cost of investing in both partitioning
and conflict traits is less than the summed costs of investing in each
trait type separately, while for the super-additive case the overall
cost is greater.

In previous models that focus on coexistence \citep[e.g.,][]{Pastore2021, 
Taper1992}, species reduce interspecific competition by evolving
traits that determine usage along a resource gradient. Evolution away
from the optimal value incurs a cost, and per capita effects of one
species on another are proportional to the difference in their
resource-uptake traits. Our model differs in that investments are
modeled only in terms of their effects on competition experienced by the
species under selection and its competitors, which encompasses
competitive interactions in a more-general way. The costs of investment
are exacted through decreases in density-independent per capita
population growth rates for both partitioning and conflict traits,
making it possible to compare the evolution of both types of traits.
Because the traits are not conceptually tied to a specific mechanism
underlying competition, the specific traits may differ among species.
For example, competition for light and apparent competition via natural
enemies could both be simultaneously included in our model. Treating
traits as abstractions allows general insights into how conflict and
partitioning traits will coevolve. Furthermore, the coevolution of
traits will determine the frequency and magnitudes of partitioning and
conflicts among species, and hence the competitive structure of
communities.

Our model includes both ecological (population abundances) and
evolutionary (trait values) variables, and therefore it is an
ecological-evolutionary model. For presenting results, we rely primarily
on simulations, although these are paralleled by analytical solutions
provided in Appendix A. We focus on two questions. What determines the
magnitude of evolutionary investment in partitioning and conflict traits
in a community? And what types of communities are the end product of
coevolution, communities dominated by species showing partitioning
traits, species showing conflict traits, or species showing both?

