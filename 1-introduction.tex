
\section*{Introduction}

Competitive communities contain a wide array of ways in which
competitors interact, and these interactions often influence
species coexistence.
Interactions that weaken interspecific competition in relation to
intraspecific competition should help stabilize coexistence,
while those doing the opposite should help destabilize it 
\citep{Chesson2000}.
A large body of work has provided many mechanisms that should stabilize
coexistence, most of which center on ecological disparities among species.
These include differences in species'
use of resources \citep{Macarthur1967a,Roughgarden1976,Abrams2009}, 
effects from natural enemies \citep{Grover1998,Vandermeer1998,Ehrlich2017a,Abrams2002},
and environmental responses
\citep{Chesson1997,Armstrong1976,Loreau1992,Chesson1994,Kremer2017,Pacala1994}.
Other factors can increase a species' per-capita effect on its competitors.
Traits that could produce these effects include
tree height \citep{Falster2003}, 
propagule size \citep{Geritz1999,Fagerstrdm2016},
or aggressive behavior \citep{Brown1971}.
Without strong coinciding tradeoffs, these mechanisms can result 
in strict hierarchies for competitive outcomes and 
can limit stable coexistence \citep{Abrams1994,Falster2003}.
Natural competitive communities likely contain a wide range of
both coexistence- and exclusion-promoting interactions,
and the effects of combinations of different interaction types
is unclear.


Evolution of ecologically relevant traits---and 
associated species interactions---adds to this complexity,
since it can have strong effects on coexistence
\citep{Wagner2017a,TerHorst2018}.
Community dynamics are affected by evolutionary change
for initial assembly \citep{Vellend2010},
subsequent invasions \citep{Faillace2016},
and long-term equilibria, both
stationary \citep{Lankau2011,Barabas2016}
and cyclical \citep{Fussmann2013,Vasseur2011a,Kremer2017}.
Evolution can contribute to coexistence by
improving invasion success \citep{Faillace2016}
or buffering against environmental perturbations 
\citep{Barabas2016,Fussmann2013,Osmond2013,Bell2017}.
More complex mechanisms of evolution mediating coexistence, 
such as synergistic pleiotropy \citep{Schreiber2018a}
or ``neighbor-dependent selection'' \citep{Vasseur2011a},
have also been demonstrated.
Alternatively, some evidence suggests that evolution can 
reduce species persistence \citep{Ferriere2013},
increase exclusion of invaders by resident species \citep{Faillace2016},
and destabilize long-term coexistence \citep{Shoresh2008}.
Recently, \citet{Pastore2021} found that evolution often caused
exclusion in a 2-species, consumer--resource model, 
largely due to evolutionary changes in competitive differences.


Here, we use a simple model to describe what happens when competitors
can evolve to reduce the effects they experience from interspecific
competition via both stabilizing and destabilizing mechanisms.
In contrast to many other models, we do not impose a specific mechanism
of coexistence (e.g., niche differences) in our model and instead focus 
on general insights that apply to many types of interactions.
In seeking generality, the model is necessarily phenomenological:
species directly affect competition experienced by themselves and
others in the community.
They do this via evolutionary investment in ``competition axes.''
Competition axes take one of two forms: 
For an ``ameliorative axis,'' investment by any species reduces 
the effects of competition on all other species.
For a ``conflicting axis,'' investment by one species
increases competitive effects for all others.
Investment also has a cost, and this cost can be either more or less
costly when a species invests in two or more axes, compared to just one.
We explore when coexistence is most likely,
and we compare these results to some of the other coexistence and exclusion
mechanisms that are often discussed.

