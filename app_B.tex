
\renewcommand{\thefigure}{B\arabic{displayFigure}}
\renewcommand{\theequation}{B\arabic{displayEquation}}
\renewcommand{\thetable}{B\arabic{displayTable}}
\setcounter{displayFigure}{0}
\setcounter{displayEquation}{0}
\setcounter{displayTable}{0}


\section*{Appendix B: General matrix forms of equations and Jacobian}


\subsection*{Matrix forms of equations}


This section defines a general form of the equations in the main text, 
for any $q$ suites of traits, using matrix notation.
We used this form for the derivatives in the subsequent section, and 
we believe that keeping them in a more general form will help make them 
more useful to those interested in building from this work.
Here, $^{\textrm{T}}$ indicates transposition,
multiplication between matrices is always matrix multiplication,
bold capital letters indicate a matrix, and bold lowercase letters indicate 
a vector.
See Table \ref{tab:matrix-parameters-table} for descriptions of the parameters.



\begin{table}[!ht]
    \centering
    \singlespacing
    \renewcommand{\arraystretch}{1.5}
    \begin{tabular}{lp{5in}}
    \toprule
    Parameter & Description \\
    \midrule
    $q$ & The number of trait suites in which species can invest. \\
    $n$ & The number of species in the community. \\
    $\mathbf{v}_i$ & A $q \times 1$ vector containing the  investments for
        species $i$. \\
    $F_i$ & Per-capita growth for species $i$. \\
    $N_i$ & Population density of species $i$. \\
    $\alpha_0$ & Base density dependence. \\
    $r_0$ & Baseline growth rate. \\
    $f$ & Cost of increasing investments on the growth rate. \\
    $\mathbf{C}$ & Contains parameters controlling non-additive costs.
        All values on the diagonal are 1. \\
    $\eta_{k,l}$ & Non-additive cost of increasing both the
        $k$\textsuperscript{th} and $l$\textsuperscript{th} suites of traits.
        When $\eta_{k,l} > 0$, investing in both suites incurs an extra cost
        compared to investing in each separately.
        When $\eta_{k,l} < 0$, investing in both suites is less costly.
        Non-additive trade-offs are symmetrical (i.e., $\eta_{k,l} = \eta_{l,k}$
        for all $l$ and $k$). \\
    $\mathbf{D}$ & Contains parameters that determine how investment in one
        species affects others. All off-diagonal elements are zero. \\
    $d_k$ & Effect on other species when one invests in the
        $k$\textsuperscript{th} suite of traits. 
        When $d_k < 0$, this investment increases competition experienced by 
        all other species (i.e., the $k$\textsuperscript{th} traits 
        are conflicting). 
        Alternatively, when $d_k > 0$, the same investment decreases the 
        effect of competition on all species (i.e., the $k$\textsuperscript{th}
        traits are partitioning). \\
    $\sigma_A^2$ & Additive genetic variance. \\
    %
    \bottomrule
    \end{tabular}
    \caption{Parameters used in the matrix equations.}
    \label{tab:matrix-parameters-table}
\end{table}


The following 4 equations correspond to Equations
% \ref{eq:pg-growth}, \ref{eq:growth-rate}, \ref{eq:competition}, and
% \ref{eq:invest-change} 
\ref{eq:pg-growth}--\ref{eq:invest-change} 
from the main text.

The per-capita growth for species $i$ is

\begin{equation} \label{eq:matrix-pg-growth}
    F_{i} = \exp \left\{ r_i(\mathbf{v}_i) - 
        \alpha_{0} \, N_i - \sum_{j \ne i}^{n}{
            \alpha_{ij}(\mathbf{v}_i, \mathbf{v}_j) \, N_j}  
    \right\}\textrm{.}
\end{equation}

The parameter $r_i(\mathbf{v}_i)$ describes the direct costs of investing to
the growth rate:

\begin{equation} \label{eq:matrix-growth-rate}
\begin{split}
    r_i(\mathbf{v}_i) &= r_0 - f \, \mathbf{v}_i^{\textrm{T}} \, 
        \mathbf{C} ~ \mathbf{v}_{i} \\
    \mathbf{C} &= 
        \begingroup % keep the changes local
        \renewcommand*{\arraystretch}{0.75}
        \setlength\arraycolsep{3pt}
        \begin{pmatrix}
        1         & \ldots & \eta_{1q} \\
        \vdots    & \ddots & \vdots \\
        \eta_{q1} & \ldots & 1      \\
        \end{pmatrix}
        \endgroup
    \textrm{.}
\end{split}
\end{equation}

\newpage{}

The term $\alpha_{ij}(\mathbf{v}_i, \mathbf{v}_j)$ in equation
\ref{eq:matrix-pg-growth}
represents how investments influence the effects
of interspecific competition:

\begin{equation} \label{eq:matrix-competition}
\begin{split}
    \alpha_{ij}(\mathbf{v}_i, \mathbf{v}_j) &= \alpha_0 ~\exp \left\{
        - \mathbf{v}_i^{\textrm{T}} \mathbf{v}_i -
        \mathbf{v}_j^{\textrm{T}} \mathbf{D} \mathbf{v}_j \right\} \\
    \mathbf{D} &= 
        \begingroup % keep the changes local
        \renewcommand*{\arraystretch}{0.75}
        \setlength\arraycolsep{3pt}
        \begin{pmatrix}
        d_1     & \ldots    & 0 \\
        \vdots  & \ddots    & \vdots \\
        0       & \ldots    & d_q
        \end{pmatrix}
        \endgroup
	\textrm{.}
\end{split}
\end{equation}


Investments at time $t+1$ are

\begin{equation} \label{eq:matrix-invest-change}
\begin{split}
    \mathbf{v}_{i,t+1} &= \mathbf{v}_{i,t} + \left( \frac{1}{F_i}
        \frac{\partial \, F_i}{\partial \, \mathbf{v}_{i,t}} \right) \sigma^2_A \\
     &= \mathbf{v}_{i,t} + 2 \sigma_A^2
    \left[
        \alpha_0 \, \left(
            \sum_{j \ne i}^{n}{ N_{j,t} \, \textrm{e}^{
            - \mathbf{v}_{j,t}^{\textrm{T}}
            \mathbf{D} \mathbf{v}_{j,t} } }
        \right) \:
            \textrm{e}^{- \mathbf{v}_{i,t}^{\textrm{T}} \mathbf{v}_{i,t}} \:
            \mathbf{v}_{i,t}^{\textrm{T}}
        - f \, \mathbf{v}_{i,t}^{\textrm{T}} \: \mathbf{C}
    \right]
    \textrm{.}
\end{split}
\end{equation}







\subsection*{Jacobian}

The Jacobian is

\begin{equation} \label{eq:jacobian}
    \begin{pmatrix}
        \frac{\partial \mathbf{v}_{1,t+1}}{\partial \mathbf{v}_{1,t}} & \cdots &
            \frac{\partial \mathbf{v}_{1,t+1}}{\partial \mathbf{v}_{n,t}} &
            \frac{\partial \mathbf{v}_{1,t+1}}{\partial N_{1,t}} & \cdots &
            \frac{\partial \mathbf{v}_{1,t+1}}{\partial N_{n,t}} \\
        \vdots & \ddots & \vdots & \vdots & \ddots & \vdots \\
        \frac{\partial \mathbf{v}_{n,t+1}}{\partial \mathbf{v}_{1,t}} & \cdots &
            \frac{\partial \mathbf{v}_{n,t+1}}{\partial \mathbf{v}_{n,t}} &
            \frac{\partial \mathbf{v}_{n,t+1}}{\partial N_{1,t}} & \cdots &
            \frac{\partial \mathbf{v}_{n,t+1}}{\partial N_{n,t}} \\[1ex]
%
%
        \frac{\partial N_{1,t+1}}{\partial \mathbf{v}_{1,t}} & \cdots &
            \frac{\partial N_{1,t+1}}{\partial \mathbf{v}_{n,t}} &
            \frac{\partial N_{1,t+1}}{\partial N_{1,t}} & \cdots &
            \frac{\partial N_{1,t+1}}{\partial N_{n,t}} \\
        \vdots & \ddots & \vdots & \vdots & \ddots & \vdots \\
        \frac{\partial N_{n,t+1}}{\partial \mathbf{v}_{1,t}} & \cdots &
            \frac{\partial N_{n,t+1}}{\partial \mathbf{v}_{n,t}} &
            \frac{\partial N_{n,t+1}}{\partial N_{1,t}} & \cdots &
            \frac{\partial N_{n,t+1}}{\partial N_{n,t}} \\
    \end{pmatrix}
    \text{.}
\end{equation}


It is made up of the following derivatives:

\begin{equation*}
\begin{split}
    \frac{ \partial \, \mathbf{v}_{i,t+1} }{ \partial \, \mathbf{v}_{i,t} } &= 
    \mathbf{I} + 2 ~ \sigma_A^2 ~
        \left[
            \alpha_0 ~ \left(
                \sum_{j \ne i}^{n}{ N_{j,t} \, \textrm{e}^{
                - \mathbf{v}_{j,t}^{\textrm{T}}
                \mathbf{D} \mathbf{v}_{j,t} } }
            \right) ~ \textrm{e}^{ - \mathbf{v}_{i,t}^{\textrm{T}} \mathbf{v}_{i,t} }
            \left(
                \mathbf{I} - 2 ~ \mathbf{v}_{i,t} \mathbf{v}_{i,t}^{\textrm{T}}
            \right) -
            f \: \mathbf{C}^{\textrm{T}}
        \right] \\\
% 
    \frac{ \partial \: \mathbf{v}_{i,t+1} }{ \partial \: \mathbf{v}_{k,t}} &=
        -4 \; \sigma_A^2 \; \alpha_0 \; N_{k,t} \; \mathbf{v}_{i,t} \;
        \textrm{e}^{
                    - \mathbf{v}_{k,t}^{\textrm{T}} \mathbf{D} \mathbf{v}_{k,t}
                    - \mathbf{v}_{i,t}^{\textrm{T}} \mathbf{v}_{i,t}
                } \;
        \mathbf{v}_{k,t}^{\textrm{T}} \; \mathbf{D} \\
% 
    \frac{ \partial \: \mathbf{v}_{i,t+1} }{ \partial \: N_{i,t} } &= 0 \\
    \frac{ \partial \: \mathbf{v}_{i,t+1} }{ \partial \: N_{k,t} } &=
        2 \; \sigma_A^2 \; \alpha_0 \; \mathbf{v}_{i,t} \;
        \textrm{e}^{ - \mathbf{v}_{k,t}^{\textrm{T}} \mathbf{D} \mathbf{v}_{k,t}
            - \mathbf{v}_{i,t}^{\textrm{T}} \mathbf{v}_{i,t} } \\
% 
    \frac{ \partial N_{i,t+1} }{ \partial \mathbf{v}_{i,t} } &= 
        2 \, F_{i,t+1} \,  N_{i,t}
        \left[
            \alpha_0 \, \left(
                \sum_{j \ne i}^{n}{ N_{j,t} \, \textrm{e}^{
                - \mathbf{v}_{j,t}^{\textrm{T}}
                \mathbf{D} \mathbf{v}_{j,t} } }
            \right) \, \text{e}^{ -\mathbf{v}_{i,t}^{\text{T}}
            \mathbf{v}_{i,t} } \, \mathbf{v}_{i,t}^{\text{T}}
            - f \, \mathbf{v}_{i,t}^{\text{T}} \, \mathbf{C}
        \right] \\
    \frac{ \partial N_{i,t+1} }{ \partial \mathbf{v}_{k,t} } &= 
        2 \, F_{i,t+1} \, N_{i,t} \, N_{k,t} \, \alpha_0 \: 
        \text{e}^{ -\mathbf{v}_{i,t}^{\text{T}} \mathbf{v}_{i,t} -
            \mathbf{v}_{k,t}^{\text{T}} \mathbf{D} \mathbf{v}_{k,t} } \:
        \mathbf{v}_{k,t}^{\text{T}} \, \mathbf{D} \\
% 
    \frac{ \partial N_{i,t+1} }{ \partial N_{i,t} } &= 
        F_{i,t+1} \left( 1 - \alpha_0 \: N_{i,t} \right) \\
    \frac{ \partial N_{i,t+1} }{ \partial N_{k,t} } &= 
        - F_{i,t+1} \: N_{i,t} \: \alpha_0 \: 
        \text{e}^{ -\mathbf{v}_{i,t}^{\text{T}} \mathbf{v}_{i,t} -
            \mathbf{v}_{k,t}^{\text{T}} \mathbf{D} \mathbf{v}_{k,t} } 
    \textrm{.}
\end{split}
\end{equation*}


