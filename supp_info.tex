

\section*{\huge{Supplemental Information}}


% You should ultimately use the supplement template offered by AmNat

\renewcommand{\thefigure}{S\arabic{figure}}
\renewcommand{\theequation}{S\arabic{equation}}
\renewcommand{\thetable}{S\arabic{table}}
\setcounter{equation}{0}
\setcounter{figure}{0}
\setcounter{table}{0}



% Note: this should go into a separate document eventually

\section*{Matrix derivatives in quantitative genetics equations}

As in the main text, $^{\textrm{T}}$ indicates transposition,
multiplication between matrices is always matrix multiplication, and
bold face indicates a matrix.
Also note that both $\mathbf{C}$ and $\mathbf{D}$ are symmetrical,
so $\mathbf{C} + \mathbf{C}^{\textrm{T}} = 2 \; \mathbf{C}$ and
$\mathbf{D} + \mathbf{D}^{\textrm{T}} = 2 \; \mathbf{D}$.




\subsection*{Jacobian matrix}

The $n(q+1) \times n(q+1)$ Jacobian matrix consists of 

\begin{itemize}
\item $n^2$ blocks of size $q \times q$ containing
    $\partial \mathbf{v}_{i,t+1} / \partial \mathbf{v}_{\zeta,t}$
\item $n^2$ blocks of size $1 \times q$ containing
    $\partial \mathbf{v}_{i,t+1} / \partial N_{\zeta,t}$
\item $n^2$ blocks of size $q \times 1$ containing
    $\partial N_{i,t+1} / \partial \mathbf{v}_{\zeta,t}$
\item $n^2$ blocks of size $1 \times 1$ containing
    $\partial N_{i,t+1} / \partial N_{\zeta,t}$
\end{itemize}


for all $i \in \{ 1, \: \ldots \, , \: n \}$
and $\zeta \in \{ 1, \: \ldots \, , \: n \}$.


The partial derivatives of species $i$ investments at time $t+1$ with respect
to species $i$ investments at time $t$ are

\begin{equation*}
\begin{split}
    \frac{ \partial \, \mathbf{v}_{i,t+1} }{ \partial \, \mathbf{v}_{i,t} } &=
        \frac{ \partial \, \mathbf{v}_{i,t} }{ \partial \, \mathbf{v}_{i,t} } +
        2 \; \sigma_A^2
        \left(
            \frac{ \partial \;
                \alpha_0 \; 
                \left( 
                    \sum_{j \ne i}^{n}{ N_{j,t} \, \textrm{e}^{
                    - \mathbf{v}_{j,t}^{\textrm{T}}
                    \mathbf{D} \mathbf{v}_{j,t} } }
                \right) \;
                    \textrm{e}^{-\mathbf{v}_{i,t}^{\textrm{T}} \mathbf{v}_{i,t}} \,
                    \mathbf{v}_{i,t}^{\textrm{T}}}{\partial \; \mathbf{v}_{i,t} } -
            \frac{ \partial \; f \, \mathbf{v}_{i,t}^{\textrm{T}} \mathbf{C}}{\partial \; \mathbf{v}_{i,t} }
        \right) \\
    &=
        \mathbf{I} +
        2 \; \sigma_A^2
        \left[
            \alpha_0
            \left( 
                \sum_{j \ne i}^{n}{ N_{j,t} \, \textrm{e}^{
                - \mathbf{v}_{j,t}^{\textrm{T}}
                \mathbf{D} \mathbf{v}_{j,t} } }
            \right) \,
            \left(
                \textrm{e}^{-\mathbf{v}_{i,t}^{\textrm{T}} \mathbf{v}_{i,t}} +
                \frac{ \partial \;
                        \textrm{e}^{-\mathbf{v}_{i,t}^{\textrm{T}} \mathbf{v}_{i,t}}
                        }{\partial \; \mathbf{v}_{i,t} } \, \mathbf{v}_{i,t}^{\textrm{T}}
            \right) -
            f \, \mathbf{C}^{\textrm{T}}
            \right] \\[2ex]
    \frac{ \partial \, \mathbf{v}_{i,t+1} }{ \partial \, \mathbf{v}_{i,t} } &= \mathbf{I} + 2 ~ \sigma_A^2 ~
        \left[
            \alpha_0 
            \left( 
                \sum_{j \ne i}^{n}{ N_{j,t} \, \textrm{e}^{
                - \mathbf{v}_{j,t}^{\textrm{T}}
                \mathbf{D} \mathbf{v}_{j,t} } }
            \right)
            \textrm{e}^{ - \mathbf{v}_{i,t}^{\textrm{T}} \mathbf{v}_{i,t} }
            \left(
                \mathbf{I} - 2 ~ \mathbf{v}_{i,t} \mathbf{v}_{i,t}^{\textrm{T}}
            \right) -
            f \: \mathbf{C}^{\textrm{T}}
        \right]
    \textrm{,}
\end{split}
\end{equation*}

\noindent where $\mathbf{I}$ is a $q \times q$ identity matrix.


Next we have the partial derivatives of species $i$ investments at time $t+1$ with respect to 
species $k$ investments at time $t$, where $k \ne i$.
To calculate this, it's useful to slightly rearrange equation \ref{eq:matrix-invest-change} to
extract the portion that includes $\mathbf{v}_{k,t}$:


\begin{equation*}
    \mathbf{v}_{i,t+1} = \mathbf{v}_{i,t} + 2 \; \sigma_A^2
    \left[
        \left(
            N_{k,t} \; \textrm{e}^{ -\mathbf{v}_{k,t}^{\textrm{T}} \mathbf{D}
            \mathbf{v}_{k,t} } + 
            \sum_{j \ne i, j \ne k}^{n}{ N_{j,t} \, \textrm{e}^{
                - \mathbf{v}_{j,t}^{\textrm{T}}
                \mathbf{D} \mathbf{v}_{j,t} } }
        \right)
        \left(
            \alpha_0 \; \textrm{e}^{-\mathbf{v}_{i,t}^{\textrm{T}}
            \mathbf{v}_{i,t} } \; \mathbf{v}_{i,t}^{\textrm{T}}
        \right)
        - f \: \mathbf{v}_{i,t}^{\textrm{T}} \mathbf{C}
    \right]
    \textrm{.}
\end{equation*}

From this we calculated the partial derivative of $\mathbf{v}_{i,t+1}$ in relation to
$\mathbf{v}_{k,t}$


\begin{equation*}
\begin{split}
    \frac{ \partial \: \mathbf{v}_{i,t+1} }{ \partial \: \mathbf{v}_{k,t} } &=
        \frac{ \partial \: \mathbf{v}_{i,t} }{ \partial \: \mathbf{v}_{k,t} } +
        2 \; \sigma_A^2 \;
        \left[
            \frac{ \partial \:
                \left(
                    N_{k,t} \textrm{e}^{- \mathbf{v}_{k,t}^{\textrm{T}} \mathbf{D}
                    \mathbf{v}_{k,t}} + 
                    \sum_{j \ne i, j \ne k}^{n}{ N_{j,t} \, \textrm{e}^{
                        - \mathbf{v}_{j,t}^{\textrm{T}}
                        \mathbf{D} \mathbf{v}_{j,t} } }
                \right)
                \left(
                    \alpha_0 \; \textrm{e}^{ - \mathbf{v}_{i,t}^{\textrm{T}}
                    \mathbf{v}_{i,t} } \mathbf{v}_{i,t}^{\textrm{T}}
                \right)
            }{ \partial \:  \mathbf{v}_{k,t} } -
            0
        \right] \\
    &= 2 \; \sigma_A^2 \; \alpha_0 \; N_{k,t} \; \mathbf{v}_{i,t} \;
        \textrm{e}^{ - \mathbf{v}_{i,t}^{\textrm{T}}
        \mathbf{v}_{i,t} } \; 
        \frac{ \partial \:
                \textrm{e}^{
                    - \mathbf{v}_{k,t}^{\textrm{T}} \mathbf{D} \mathbf{v}_{k,t}
                    }
            }{ \partial \:  \mathbf{v}_{k,t} } \\
    &= 2 \; \sigma_A^2 \; \alpha_0 \; N_{k,t} \; \mathbf{v}_{i,t} \;
        \textrm{e}^{
                    - \mathbf{v}_{k,t}^{\textrm{T}} \mathbf{D} \mathbf{v}_{k,t}
                    - \mathbf{v}_{i,t}^{\textrm{T}} \mathbf{v}_{i,t}
                } \;
        \left[ 
            - 2 \, \mathbf{v}_{k,t}^{\textrm{T}} \, \mathbf{D}
        \right] \\
    \frac{ \partial \: \mathbf{v}_{i,t+1} }{ \partial \: \mathbf{v}_{k,t}} &=
        -4 \; \sigma_A^2 \; \alpha_0 \; N_{k,t} \; \mathbf{v}_{i,t} \;
        \textrm{e}^{
                    - \mathbf{v}_{k,t}^{\textrm{T}} \mathbf{D} \mathbf{v}_{k,t}
                    - \mathbf{v}_{i,t}^{\textrm{T}} \mathbf{v}_{i,t}
                } \;
        \mathbf{v}_{k,t}^{\textrm{T}} \; \mathbf{D}
    \textrm{.} \\
\end{split}
\end{equation*}

The partial derivatives of species $i$ investments in relation to species $i$ 
and species $k$ abundances are

\begin{equation*}
\begin{split}
    \frac{ \partial \: \mathbf{v}_{i,t+1} }{ \partial \: N_{i,t} } &= 0 \\
    \frac{ \partial \: \mathbf{v}_{i,t+1} }{ \partial \: N_{k,t} } &=
        2 \; \sigma_A^2 \; \alpha_0 \; \mathbf{v}_{i,t} \;
        \textrm{e}^{ - \mathbf{v}_{k,t}^{\textrm{T}} \mathbf{D} \mathbf{v}_{k,t}
            - \mathbf{v}_{i,t}^{\textrm{T}} \mathbf{v}_{i,t} }
    \textrm{.} \\
\end{split}
\end{equation*}




For partial derivatives of abundances in relation to the other state variables,
we can combine equations from the Methods:

\begin{equation} \label{eq:fitness-full}
\begin{split}
    N_{i,t+1} &= N_{i,t} \; F_{i,t+1} \\
    F_{i,t+1} &= \exp \left\{
        r_0 - f \; \mathbf{v}_{i,t}^{\textrm{T}} \; \mathbf{C} \; \mathbf{v}_{i,t} -
        \alpha_0 \, N_{i,t} -
        \alpha_0 \;\textrm{e}^{- \mathbf{v}_{i,t}^{\textrm{T}} \mathbf{v}_{i,t} } \left( 
            \sum_{j \ne i}^{n}{ N_{j,t} \, \textrm{e}^{
                - \mathbf{v}_{j,t}^{\textrm{T}}
                \mathbf{D} \mathbf{v}_{j,t} } }
        \right)
        \right\}
    \textrm{.}
\end{split}
\end{equation}





Using these, we have the following derivatives of abundances in relation
to investments:

\begin{equation*}
\begin{split}
    \frac{ \partial N_{i,t+1} }{ \partial \mathbf{v}_{i,t} } &= 
        2 \, F_{i,t+1} \,  N_{i,t}
        \left[
            \alpha_0 \, 
            \left( 
                \sum_{j \ne i}^{n}{ N_{j,t} \, \textrm{e}^{
                    - \mathbf{v}_{j,t}^{\textrm{T}}
                    \mathbf{D} \mathbf{v}_{j,t} } }
            \right)
            \, \text{e}^{ -\mathbf{v}_{i,t}^{\text{T}}
            \mathbf{v}_{i,t} } \, \mathbf{v}_{i,t}^{\text{T}}
            - f \, \mathbf{v}_{i,t}^{\text{T}} \, \mathbf{C}
        \right] \\
    \frac{ \partial N_{i,t+1} }{ \partial \mathbf{v}_{k,t} } &= 
        2 \, F_{i,t+1} \, N_{i,t} \, N_{k,t} \, \alpha_0 \: 
        \text{e}^{ -\mathbf{v}_{i,t}^{\text{T}} \mathbf{v}_{i,t} -
            \mathbf{v}_{k,t}^{\text{T}} \mathbf{D} \mathbf{v}_{k,t} } \:
        \mathbf{v}_{k,t}^{\text{T}} \, \mathbf{D}
    \textrm{.}
\end{split}
\end{equation*}

We also have the following derivatives of abundances at time $t+1$ in relation
to those at time $t$:

\begin{equation*}
\begin{split}
    \frac{ \partial N_{i,t+1} }{ \partial N_{i,t} } &= 
        F_{i,t+1}
        \left(
            1 - \alpha_0 \: N_{i,t} 
        \right) \\
    %
    \frac{ \partial N_{i,t+1} }{ \partial N_{k,t} } &= 
        - F_{i,t+1} \: N_{i,t} \: \alpha_0 \: 
        \text{e}^{ -\mathbf{v}_{i,t}^{\text{T}} \mathbf{v}_{i,t} -
            \mathbf{v}_{k,t}^{\text{T}} \mathbf{D} \mathbf{v}_{k,t} } 
    \textrm{.}
\end{split}
\end{equation*}











% ==============================================================================
% ==============================================================================
% ==============================================================================
% ==============================================================================
% ==============================================================================
% ==============================================================================
% ==============================================================================


\section*{Effects of stochasticity on investment evolution}

Here, I am showing the equations just for the exclusionary suite of traits
(i.e., $x$), but the math is exactly the same for $p$.

From the main text, we know that when there is stochasticity in 
investment evolution, 

\begin{equation*}
\begin{split}
    \ddot{x}_{i,t+1} &= x_{i,t+1} \; \text{e}^{\varepsilon_x} \\
    \varepsilon_x &\sim \text{N}(0, \, \sigma^2_{x}) \\
    x_{i,t+1} &= x_{i,t} + \left( \frac{1}{F_i}
        \frac{\partial F_i}{\partial \ddot{x}_{i,t}} \right) \sigma_A^2 \, \text{e}^{\varepsilon_x}
    \text{.}
\end{split}
\end{equation*}



We can define the function $G_x$ for exclusionary traits as a function 
of the random variables $\varepsilon_{x}$ and $\varepsilon_{p}$:

\begin{equation*}
\begin{split}
    G_x(\varepsilon_{x}, \varepsilon_{p}) &= x_{i,t} + 2 \; \sigma_A^2 \, \text{e}^{\varepsilon_x}
    \Bigg[
        \alpha_0
            \Omega_i \;
            \text{e}^{-(x_{i,t} \text{e}^{\varepsilon_{x}})^2 - (p_{i,t} \text{e}^{\varepsilon_{p}})^2} \; x_{i,t} \, 
            \text{e}^{\varepsilon_{x}}
        - f \; ( x_{i,t} \, \text{e}^{\varepsilon_{x}} + \eta \; p_{i,t} \,
            \text{e}^{\varepsilon_{p}} )
    \Bigg] \\
    \Omega_i &\equiv \sum_{j \ne i}^{n}{ N_{j,t} \, \textrm{e}^{
                    - d_x \ddot{x}_{j,t} - d_p \ddot{p}_{j,t} } }
    \textrm{.}
\end{split}
\end{equation*}

Here, $\Omega_i$ is the influence of other species on the effect of
interspecific competition,
which is defined here to reduce clutter in later equations.
The second order Taylor series approximation for $G_x(\varepsilon_{x}, \varepsilon_{p})$,
about $\bm{\theta} = \{ \mu_{\varepsilon_{x}}, \mu_{\varepsilon_{p}} \} = \{ 0, 0 \}$, is

\begin{equation}
\label{eq:taylor-expansion-outline}
    \text{E}(G_x(\varepsilon_{x}, \varepsilon_{p})) \approx G_x(\bm{\theta}) + 
        \frac{1}{2} \left[ 
            \left. \frac{\partial^2 G_x}{\partial^2 \varepsilon_{x}} \right\lvert_{\bm{\theta}} \; \sigma^2_{x} +
            \left. \frac{\partial^2 G_x}{\partial^2 \varepsilon_{p}} \right\lvert_{\bm{\theta}} \; \sigma^2_{p}
        \right]
\text{.}
\end{equation}


In the derivatives below, we have removed indices for species and time for
clarity.
We have the following derivatives of $G_x$ with respect to the random variables
$\varepsilon_{x}$ and $\varepsilon_{p}$:

\begin{equation*}
\begin{split}
    \frac{\partial \, G_x}{\partial \, \varepsilon_{x}} &= 2 \; \sigma_A^2 \: x \: \text{e}^{\varepsilon_{x}}
    \left[
        \alpha_0 \; \Omega \;
            \text{e}^{-(x \: \text{e}^{\varepsilon_{x}})^2 - (p \: \text{e}^{\varepsilon_{p}})^2}
            \left(
                1 - 2 \, x^2 \: \text{e}^{2 \varepsilon_{x}}
            \right)
        - f
    \right] \\
% 
    \frac{\partial^2 G_x}{\partial^2 \varepsilon_{x}} &= 2 \; \sigma_A^2 \: x \: \text{e}^{\varepsilon_{x}}
    \left[
        \alpha_0 \; \Omega \;
            \text{e}^{-(x \: \text{e}^{\varepsilon_{x}})^2 - (p \: \text{e}^{\varepsilon_{p}})^2}
            \left(
                1 - 8 \, x^2 \: \text{e}^{2 \varepsilon_{x}}
            \right)
        - f
    \right] \\[2ex]
%
%
    \frac{\partial \, G_x}{\partial \, \varepsilon_{p}} &= - 2 \; \sigma_A^2
    \left[
        2 \: \alpha_0 \; \Omega \; x \; p^2 \;
            \text{e}^{-(x \: \text{e}^{\varepsilon_{x}})^2 - (p \: \text{e}^{\varepsilon_{p}})^2 + \varepsilon_{x} + 2 \, \varepsilon_{p}}
        + f \: \eta \: p \: \text{e}^{\varepsilon_{p}}
    \right] \\
% 
    \frac{\partial^2 G_x}{\partial^2 \varepsilon_{p}} &= - 2 \; \sigma_A^2
    \left[
        4 \: \alpha_0 \; \Omega \; x \; p^2 \;
            \text{e}^{-(x \: \text{e}^{\varepsilon_{x}})^2 - (p \: \text{e}^{\varepsilon_{p}})^2 + \varepsilon_{x} + 2 \, \varepsilon_{p}}
            \left(
                1 - p^2 \: \text{e}^{2 \varepsilon_{p}}
            \right)
        + f \: \eta \: p \: \text{e}^{\varepsilon_{p}}
    \right]
\text{.}
\end{split}
\end{equation*}


Combining these with equation \ref{eq:taylor-expansion-outline} give us 

\begin{equation*}
\begin{split}
    \text{E}(G_x(\varepsilon_{x}, \varepsilon_{p})) \approx \; &
        x + 2 \; \sigma_A^2 \left[ 
            \alpha_0 \; \Omega \; x \; \text{e}^{-x^2 - p^2} - f (x + \eta \; p)
        \right] \\
        &+ \frac{1}{2} \Bigg\{
            2 \; \sigma_A^2 \; x \left[
                \alpha_0 \; \Omega \; \text{e}^{-x^2 - p^2} \left( 1 - 8 \; x^2 \right) - f
            \right] \sigma^2_{x} \\
            & \hspace{2.5em} - 2 \; \sigma_A^2 \left[
                4 \; \alpha_0 \; \Omega \; x \; p^2 \; \text{e}^{-x^2 - p^2} \left( 1 - p^2 \right) + f \; \eta \; p
            \right] \sigma^2_{p}
        \Bigg\} %\\[2ex]
%
%
%    \approx ~ & x + \sigma_A^2 \Big[ 
%            2 \alpha_0 \Omega x \text{e}^{-x^2 - p^2} + 2 f (x + \eta p)
%            + \\ & \hspace*{4em} \sigma_{\varepsilon_{x}}^2 x \alpha_0 \Omega \text{e}^{-x^2 - p^2} (1 - 8 x^2) + \sigma_{\varepsilon_{x}}^2 x f
%            - \\ & \hspace*{4em} \sigma_{\varepsilon_{p}}^2 4 \alpha_0 \Omega x p^2 \text{e}^{-x^2 - p^2} (1 - p^2) - \sigma_{\varepsilon_{p}}^2 f \eta p^2
%        \Big] \\[2ex]
%
%
\text{.}
\end{split}
\end{equation*}

This simplifies to 

\begin{equation}
\label{eq:taylor-expansion-final-supp}
\begin{split}
    \text{E}(G_x(\varepsilon_{x}, \varepsilon_{p})) \approx
        x + 2 \; \sigma_A^2 \Bigg\{ 
            & \alpha_0 \; \Omega \; x \; \text{e}^{-x^2 - p^2} 
            \color{red} \bigg[ 
                1 + \sigma^2_{x} \left( \frac{1}{2} - 4 \; x^2 \right)
                - 2 \; \sigma^2_{p} \, p^2 \left( 1 - p^2 \right)
            \bigg] \\
            & - f \bigg[
                x + \eta \; p 
                {\color{red} + \frac{1}{2} \left(
                    \sigma^2_{x} \, x + \sigma^2_{p} \, \eta \; p
                \right)}
            \bigg]
        \Bigg\}
\text{.}
\end{split}
\end{equation}


Red text indicates parts not present in the non-stochastic version. 








% ========================================================================================
% ========================================================================================
% ========================================================================================
% ========================================================================================
% ========================================================================================
% ========================================================================================
% ========================================================================================

% \clearpage

% \section*{Supplemental figures}



