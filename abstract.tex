\section*{Abstract}


% Should be < 200 words (150 for note) for AmNat


The long history of research on coevolving, competitive communities has resulted in 
many mechanisms that affect coexistence among species.
Here, we provide general insights into the process of coexistence that span
many of these mechanisms.
We did this using a simple model of competitive communities where 
species can evolve to reduce competition by increasing
``competition investment axes.''
Each axis represents an amalgam of multiple traits that similarly decrease
competition as species increase its value.
We found that investment tradeoffs and whether evolution is conflicting
(i.e., investment by one species increases the effects of competition 
on others) together determine the rule(s) of coexistence for a system.
These can be a single rule across the entire axis space, or
a set of rules that depend on the axis values present in the community.
In the latter case, starting conditions can drastically affect the outcomes.
Stochasticity had both positive and negative effects on coexistence:
It reduced effects of resident species on new invaders, 
but increased the base rate of extinction.
Deterministic rules were robust to these effects in some cases, but
they broke down completely in others.
Our results help us understand how the general properties underlying
competition affect coexistence among coevolving species.



