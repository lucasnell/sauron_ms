\section*{Abstract}


% Should be < 200 words (150 for note) for AmNat


The long history of research on coevolving, competitive communities has resulted in 
many mechanisms that affect coexistence among species.
Here, we provide general insights into the process of coexistence that span
many of these mechanisms.
We did this using both deterministic and stochastic simulations of a
simple model of competitive communities where species can evolve to reduce 
competition by investing in ``competition axes.''
Each axis represents an amalgam of multiple traits that similarly decrease
competition as species increase its value.
Species have at least two axes:
a conflicting axis, where investment by one species increases the effects of
competition on others, and 
an ameliorative axis, where investment decreases competition for all species.
Deterministic simulations show that coexistence cannot occur when tradeoffs are 
sub-additive (investing in multiple axes is less costly than the sum cost of 
investing in them separately) and the strength of the
ameliorative axis is less than that for the conflicting axis.
Coexistence also cannot occur when tradeoffs are 
super-additive or additive and just one species invests more in the conflicting axis
than the ameliorative axis.
Adding stochasticity in axis evolution caused a switch for additive tradeoffs, 
where the system behaved more like when there were sub-additive than super-additive 
tradeoffs.
Stochasticity also affected coexistence via transient effects early during
species invasions.
Over most of the axis space, stochasticity had a negative effect
on probability of coexistence, but stochasticity allowed for coexistence
in marginal areas immediately outside where deterministic rules
would dictate exclusion of invaders.
Our results help us understand how the general properties underlying
competition affect coexistence among coevolving species.



