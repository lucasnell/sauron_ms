\section*{Appendix C: Supplemental Figures}

\renewcommand{\thefigure}{C\arabic{figure}}
\renewcommand{\theequation}{C\arabic{equation}}
\renewcommand{\thetable}{C\arabic{table}}
\setcounter{equation}{0}
\setcounter{figure}{0}
\setcounter{table}{0}



\begin{figure}[ht!]
\centering
\includegraphics{S1-coexist.pdf}
\caption{Number of surviving species for 24 simulations of 2-trait 
    communities
    (A) after 50,000 generations for all permutations of the traits
    being conflicting (``$-$'') or non-conflicting (``$+$''),
    (B) after 50,000 generations for varying values of $d_2$ when 
    $d_1$ is kept positive (i.e., trait 1 is kept non-conflicting), and
    (C) through time with $d_2 = -10^{-2}$ and $d_2 = -10^{-4}$.
    For all panels, $d_1 = 0.05$, $\eta = 0.6$, and species start with
    random trait values ($\sim \text{N}(0,2)$ truncated $> 0$).
    In (A), $d_2 = 0.1$.}
\label{fig:coexist}
\end{figure}


\begin{figure}[ht!]
\centering
\includegraphics{S2-invasion.pdf}
\caption{Invasion success for a 2-trait equilibrium community based on
    the invader's starting distance from the equilibrium point in trait space.
    Sub-panel columns indicate the type of evolution.
    Sub-panels rows indicate the invaders' starting
    abundances ($N_{eq}$) in relation to the residents'
    abundances ($N_{res}$); all residents had the same
    abundance.
    Point color indicates the size of the resident community.
    In these simulations, the resident community all had
    trait values based on the analytical solutions for equilibria
    in Appendix B.
    The single invading species always had its first trait start
    at the equilibrium value, but its second trait 
    varied from
    $-0.8$ to $0.8$ from the equilibrium value.
    Simulations ran for 50,000 generations and assessed invasion 
    success as the presence of the invader.
    Here, $\eta = -0.6$, $d \in \{ -0.01, \; 0, \; 0.01 \}$.
}
\label{fig:invasion}
\end{figure}


