\section*{Appendix A: Matrix derivatives in quantitative genetics equations}

\renewcommand{\thefigure}{A\arabic{figure}}
\renewcommand{\theequation}{A\arabic{equation}}
\renewcommand{\thetable}{A\arabic{table}}
\setcounter{equation}{0}
\setcounter{figure}{0}
\setcounter{table}{0}


As in the main text, $^{\textrm{T}}$ indicates transposition,
multiplication between matrices is always matrix multiplication, and
bold face indicates a matrix.
Also note that both $\mathbf{C}$ and $\mathbf{D}$ are symmetrical,
so $\mathbf{C} + \mathbf{C}^{\textrm{T}} = 2 \; \mathbf{C}$ and
$\mathbf{D} + \mathbf{D}^{\textrm{T}} = 2 \; \mathbf{D}$.


\subsection*{Trait change}

From the main text, we know that

\begin{equation*}
\begin{split}
    F_{i,t+1} &= \exp \left\{
        r_0 - f \; \mathbf{V}_{i,t}^{\textrm{T}} \; \mathbf{C} \; \mathbf{V}_{i,t} -
        \alpha_0 \;\textrm{e}^{- \mathbf{V}_{i,t}^{\textrm{T}} \mathbf{V}_{i,t} } \mathbf{\Omega}_{i,t}
        \right\} \\
    \mathbf{V}_{i,t+1} &= \mathbf{V}_{i,t} + \left( \frac{1}{F_{i,t+1}}
        \frac{\partial F_{i,t+1}}{\partial \mathbf{V}_{i,t}} \right) \sigma^2_i
    \textrm{.}
\end{split}
\end{equation*}


The partial derivative of fitness in relation to traits for species $i$ is


\begin{equation*}
\begin{split}
    \frac{\partial F_{i,t+1}}{\partial \mathbf{V}_{i,t}} &=
        \exp \left\{
            r_0
            - f \mathbf{V}_{i,t}^{\textrm{T}} \mathbf{C} \mathbf{V}_{i,t}
            - \alpha_0  \mathbf{\Omega}_{i,t} \,
                \textrm{e}^{- \mathbf{V}_{i,t}^{\textrm{T}} \mathbf{V}_{i,t}}
        \right\}
        \frac{\partial \!
            \left(
                r_0
                - f \; \mathbf{V}_{i,t}^{\textrm{T}} \; \mathbf{C} \; \mathbf{V}_{i,t}
                - \alpha_0 \; \mathbf{\Omega}_{i,t} \;
                    \textrm{e}^{- \mathbf{V}_{i,t}^{\textrm{T}} \mathbf{V}_{i,t}}
            \right)
            }{ \partial \mathbf{V}_{i,t} } \\
     &=
        \exp \left\{
            r_0
            - f \mathbf{V}_{i,t}^{\textrm{T}} \mathbf{C} \mathbf{V}_{i,t}
            - \alpha_0  \mathbf{\Omega}_{i,t} \,
                \textrm{e}^{- \mathbf{V}_{i,t}^{\textrm{T}} \mathbf{V}_{i,t}}
        \right\}
        \left[
            - 2 f \mathbf{V}_{i,t}^{\textrm{T}} \mathbf{C}
            - \alpha_0 \, \mathbf{\Omega}_{i,t} \,
                \textrm{e}^{- \mathbf{V}_{i,t}^{\textrm{T}} \mathbf{V}_{i,t} } \:
                \frac{\partial \! \left( - \mathbf{V}_{i,t}^{\textrm{T}} \mathbf{V}_{i,t} \right)
                    }{ \partial \mathbf{V}_{i,t} }
        \right] \\[2ex]
    \frac{ \partial F_{i,t} }{ \partial \mathbf{V}_{i,t} } &=
        \exp \left\{
            r_0
            - f \mathbf{V}_{i,t}^{\textrm{T}} \mathbf{C} \mathbf{V}_{i,t}
            - \alpha_0  \mathbf{\Omega}_{i,t} \,
                \textrm{e}^{- \mathbf{V}_{i,t}^{\textrm{T}} \mathbf{V}_{i,t}}
        \right\}
        \left[
            2 \alpha_0 \mathbf{\Omega}_{i,t} \,
                \textrm{e}^{- \mathbf{V}_{i,t}^{\textrm{T}} \mathbf{V}_{i,t}} \:
                \mathbf{V}_{i,t}^{\textrm{T}}
            - 2 f \mathbf{V}_{i,t}^{\textrm{T}} \mathbf{C}
        \right]
    \textrm{.}
\end{split}
\end{equation*}



Combining above with equation \ref{eq:trait-change}, we find that trait values at
time $t+1$ are

\begin{equation} \label{eq:trait-change-full}
    \mathbf{V}_{i,t+1} = \mathbf{V}_{i,t} + 2 \sigma_i^2
    \left(
        \alpha_0 \, \mathbf{\Omega}_{i,t} \:
            \textrm{e}^{- \mathbf{V}_{i,t}^{\textrm{T}} \mathbf{V}_{i,t}} \:
            \mathbf{V}_{i,t}^{\textrm{T}}
        - f \, \mathbf{V}_{i,t}^{\textrm{T}} \: \mathbf{C}
    \right)
    \textrm{.}
\end{equation}


\subsection*{Jacobian matrix}

The $n(q+1) \times n(q+1)$ Jacobian matrix consists of 

\begin{itemize}
\item $n^2$ blocks of size $q \times q$ containing
    $\partial \mathbf{V}_{i,t+1} / \partial \mathbf{V}_{\zeta,t}$
\item $n^2$ blocks of size $q \times 1$ containing
    $\partial \mathbf{V}_{i,t+1} / \partial N_{\zeta,t}$
\item $n^2$ blocks of size $1 \times q$ containing
    $\partial N_{i,t+1} / \partial \mathbf{V}_{\zeta,t}$
\item $n^2$ blocks of size $1 \times 1$ containing
    $\partial N_{i,t+1} / \partial N_{\zeta,t}$
\end{itemize}


for all $i \in \{ 1, \: \ldots \, , \: n \}$
and $\zeta \in \{ 1, \: \ldots \, , \: n \}$.


The partial derivatives of species $i$ traits at time $t+1$ with respect
to species $i$ traits at time $t$ are

\begin{equation*}
\begin{split}
    \frac{ \partial \, \mathbf{V}_{i,t+1} }{ \partial \, \mathbf{V}_{i,t} } &=
        \frac{ \partial \, \mathbf{V}_{i,t} }{ \partial \, \mathbf{V}_{i,t} } +
        2 \; \sigma_i^2
        \left(
            \frac{ \partial \;
                \alpha_0 \; \mathbf{\Omega}_{i,t} \;
                    \textrm{e}^{-\mathbf{V}_{i,t}^{\textrm{T}} \mathbf{V}_{i,t}} \,
                    \mathbf{V}_{i,t}^{\textrm{T}}}{\partial \; \mathbf{V}_{i,t} } -
            \frac{ \partial \; f \, \mathbf{V}_{i,t}^{\textrm{T}} \mathbf{C}}{\partial \; \mathbf{V}_{i,t} }
        \right) \\
    &=
        \mathbf{I} +
        2 \; \sigma_i^2
        \left[
            \alpha_0 \; \mathbf{\Omega}_{i,t} \,
            \left(
                \textrm{e}^{-\mathbf{V}_{i,t}^{\textrm{T}} \mathbf{V}_{i,t}} +
                \frac{ \partial \;
                        \textrm{e}^{-\mathbf{V}_{i,t}^{\textrm{T}} \mathbf{V}_{i,t}}
                        }{\partial \; \mathbf{V}_{i,t} } \, \mathbf{V}_{i,t}^{\textrm{T}}
            \right) -
            f \, \mathbf{C}^{\textrm{T}}
            \right] \\[2ex]
    \frac{ \partial \, \mathbf{V}_{i,t+1} }{ \partial \, \mathbf{V}_{i,t} } &= \mathbf{I} + 2 ~ \sigma_i^2 ~
        \left[
            \alpha_0 ~ \mathbf{\Omega}_{i,t} ~ \textrm{e}^{ - \mathbf{V}_{i,t}^{\textrm{T}} \mathbf{V}_{i,t} }
            \left(
                \mathbf{I} - 2 ~ \mathbf{V}_{i,t} \mathbf{V}_{i,t}^{\textrm{T}}
            \right) -
            f \: \mathbf{C}^{\textrm{T}}
        \right]
    \textrm{,}
\end{split}
\end{equation*}

\noindent where $\mathbf{I}$ is a $q \times q$ identity matrix.


Next we have the partial derivatives of species $i$ traits at time $t+1$ with respect to 
species $k$ traits at time $t$, where $k \ne i$.
To calculate this, it's useful to rearrange equation \ref{eq:trait-change-full} and
extract the portion that includes $\mathbf{V}_{k,t}$:

% >>>>>>>>>>>>>>>>>>>>>>>>>>>>>>>>>>>>>>>>>>>>>>>>>>>>>>>>>>>>>>>>>>>>>>>>>>>
% >>>>>>>>>>>>>>>>>>>>>>>>>>>>>>>>>>>>>>>>>>>>>>>>>>>>>>>>>>>>>>>>>>>>>>>>>>>

% LEFT OFF HERE

% >>>>>>>>>>>>>>>>>>>>>>>>>>>>>>>>>>>>>>>>>>>>>>>>>>>>>>>>>>>>>>>>>>>>>>>>>>>
% >>>>>>>>>>>>>>>>>>>>>>>>>>>>>>>>>>>>>>>>>>>>>>>>>>>>>>>>>>>>>>>>>>>>>>>>>>>


\begin{equation*}
\begin{split}
    \mathbf{V}_{i,t+1} &= \mathbf{V}_{i,t} + 2 \; \sigma_i^2
    \left[
        \left(
            N_{k,t} \; \textrm{e}^{-\mathbf{V}_{k,t} \mathbf{D} \mathbf{V}_{k,t}^\textrm{T}} +
            \mathbf{\Phi}_{i,t}
        \right)
        \left(
            \alpha_0 \; \textrm{e}^{-\mathbf{V}_{i,t}
            \mathbf{V}_{i,t}^\textrm{T}} \; \mathbf{V}_{i,t}
        \right)
        - f \mathbf{V}_{i,t} \mathbf{C}
    \right] \\
    \mathbf{\Phi}_{i,t} &= N_{i,t} + \sum_{j \ne i, j \ne k}^{n}{
        N_{j,t} \; \textrm{e}^{- \mathbf{V}_{j,t} \mathbf{D}
        \mathbf{V}_{j,t}^{\textrm{T}}} }
    \textrm{.}
\end{split}
\end{equation*}

From this we calculated the partial derivative of $\mathbf{V}_{i,t+1}$ in relation to
$\mathbf{V}_{k,t}$


\begin{equation*}
\begin{split}
    \frac{ \partial \: \mathbf{V}_{i,t+1} }{ \partial \: \mathbf{V}_{k,t} } &=
        \frac{ \partial \: \mathbf{V}_{i,t} }{ \partial \: \mathbf{V}_{k,t} } +
        2 \; \sigma_i^2 \;
        \left[
            \frac{ \partial \:
                \left(
                    N_{k,t} \textrm{e}^{- \mathbf{V}_{k,t} \mathbf{D}
                    \mathbf{V}_{k,t}^{\textrm{T}}} + \mathbf{\Phi}_{i,t}
                \right)
                \left(
                    \alpha_0 \; \textrm{e}^{- \mathbf{V}_{i,t}
                    \mathbf{V}_{i,t}^{\textrm{T}}} \mathbf{V}_{i,t}
                \right)
            }{ \partial \:  \mathbf{V}_{k,t} } -
            \frac{ \partial \:  f \, \mathbf{V}_{i,t} \mathbf{C} }{
            \partial \: \mathbf{V}_{k,t} }
        \right] \\
    &= 2 \; \sigma_i^2 \; \alpha_0 \; N_{k,t} \;
        \frac{ \partial \:
                \textrm{e}^{
                    - \mathbf{V}_{k,t} \mathbf{D} \mathbf{V}_{k,t}^{\textrm{T}}
                    - \mathbf{V}_{i,t} \mathbf{V}_{i,t}^{\textrm{T}}
                } \; \mathbf{V}_{i,t}
            }{ \partial \:  \mathbf{V}_{k,t} } \\
    &= 2 \; \sigma_i^2 \; \alpha_0 \; N_{k,t} \;
        \frac{ \partial \:
                \left(
                    - \mathbf{V}_{k,t} \mathbf{D} \mathbf{V}_{k,t}^{\textrm{T}}
                    - \mathbf{V}_{i,t} \mathbf{V}_{i,t}^{\textrm{T}}
                \right)
            }{ \partial \:  \mathbf{V}_{k,t} } \;
        \textrm{e}^{
                    - \mathbf{V}_{k,t} \mathbf{D} \mathbf{V}_{k,t}^{\textrm{T}}
                    - \mathbf{V}_{i,t} \mathbf{V}_{i,t}^{\textrm{T}}
                } \; \mathbf{V}_{i,t} \\
    &= - 2 \; \sigma_i^2 \; \alpha_0 \; N_{k,t} \,
        \left( \mathbf{D} + \mathbf{D}^{\textrm{T}} \right) \,
        \mathbf{V}_{k,t}^{\textrm{T}} \;
        \textrm{e}^{
                    - \mathbf{V}_{k,t} \mathbf{D} \mathbf{V}_{k,t}^{\textrm{T}}
                    - \mathbf{V}_{i,t} \mathbf{V}_{i,t}^{\textrm{T}}
                } \; \mathbf{V}_{i,t} \\[2ex]
    \frac{ \partial \: \mathbf{V}_{i,t+1} }{ \partial \: \mathbf{V}_{k,t}} &=
        -4 \; \sigma_i^2 \; \alpha_0 \; N_{k,t} \;
        \mathbf{D} \; \mathbf{V}_{k,t}^{\textrm{T}} \;
        \textrm{e}^{
                - \mathbf{V}_{k,t} \mathbf{D} \mathbf{V}_{k,t}^{\textrm{T}}
                - \mathbf{V}_{i,t} \mathbf{V}_{i,t}^{\textrm{T}}
            } \;
            \mathbf{V}_{i,t}
    \textrm{.} \\
\end{split}
\end{equation*}



\subsection*{Keeping traits non-negative}


The previous section in this appendix shows the equations for the
partial derivatives when traits are allowed to be $<0$.
This section shows how those derivatives are altered to account for
the fact that traits are kept $\ge 0$.


An obvious choice for keeping traits from being $<0$ is using absolute
values.
However, during simulations, using absolute values causes fluctuations
in the trait values when they approach zero (they ``bounce off''
the zero bound) that persist for a very long time.
This causes the simulations to take a prohibitively long time to reach
equilibrium.
A more important disadvantage is that $d \lvert x \rvert / dx$ is
undefined when $x = 0$.


Instead we are using the ramp function, $R(x)$, which is defined as

\begin{equation*}
    R(x) = \begin{cases}
        x & \text{if}\ x \ge 0 \\
        0 & \text{if}\ x < 0
        \end{cases}
    \text{.}
\end{equation*}


\noindent Its derivative is

\begin{equation*}
    R'(x) = H(x).
\end{equation*}

\noindent $H(x)$ is called the ``Heaviside step function'' (Weisstein n.d.):

\begin{equation*}
    H(x) = \begin{cases}
        0 & \text{if}\ x \le 0 \\
        1 & \text{if}\ x > 0
        \end{cases}
    \text{.}
\end{equation*}

% Also see https://see.stanford.edu/materials/lsoftaee261/book-fall-07.pdf


To show how we altered the equations for the partial derivatives
in the previous section, it's useful to define a general form of how
the equations for trait change through time are altered to keep traits $\ge 0$.
Any one trait value at time $t+1$,
$v_{ij,t+1}$ (for species $i$ and trait $j$),
is a function of any trait value at time $t$, $v_{\zeta\psi,t}$
(where $\zeta \in \{ 1, \: \ldots \, , \: n \}$,
$\psi \in \{ 1, \: \ldots \, , \: q \}$)
as follows:

\begin{equation*}
    v_{ij,t+1}(v_{\zeta\psi,t}) = R(\ddot{v}_{ij,t+1}(v_{\zeta\psi,t}))
\end{equation*}

\noindent where $\ddot{v}_{ij,t+1}(v_{\zeta\psi,t})$ is the form of the equation that
allows negative values (i.e., those in the section above).
For the partial derivative in relation to $v_{\zeta\psi,t}$, we get


\begin{equation*}
    \frac{\partial \, v_{ij,t+1}(v_{\zeta\psi,t})}{\partial \, v_{\zeta\psi,t}} =
        H(\ddot{v}_{ij,t+1}(v_{\zeta\psi,t}))
        \frac{\partial \, \ddot{v}_{ij,t+1}(v_{\zeta\psi,t})}{
            \partial \, v_{\zeta\psi,t} }
\text{.}
\end{equation*}

\noindent where
$\frac{\partial \, \ddot{v}_{ij,t+1}(v_{\zeta\psi,t})}{
\partial \, v_{\zeta\psi,t}}$
is one of the partial derivatives described in the previous section
(which one depends on whether $i = \zeta$).


Thus, by including the ramp function, we must multiply the derivatives
calculated using the equations in the previous section by the term
$H(\ddot{v}_{ij,t+1}(v_{\zeta\psi,t}))$ to get the updated derivative.
Based on the definition of $H(x)$...

\begin{itemize}
    \item If $\ddot{v}_{ij,t+1}(v_{\zeta\psi,t})$ is $> 0$ at equilibrium, then
        $\frac{\partial \, v_{ij,t+1}(v_{\zeta\psi,t})}{
            \partial \, v_{\zeta\psi,t}}$
        simplifies to $\frac{\partial \, \ddot{v}_{ij,t+1}(v_{\zeta\psi,t})}{
            \partial \, v_{\zeta\psi,t}}$.
    \item If $\ddot{v}_{ij,t+1}(v_{\zeta\psi,t})$ is $\le 0$ at equilibrium, then
        $\frac{\partial \, v_{ij,t+1}(v_{\zeta\psi,t})}{
            \partial \, v_{\zeta\psi,t}}$
        simplifies to zero.
\end{itemize}




\subsection*{Reference}

Weisstein, Eric W. n.d. Ramp Function.
From MathWorld--A Wolfram Web Resource.
\url{https://mathworld.wolfram.com/RampFunction.html}
