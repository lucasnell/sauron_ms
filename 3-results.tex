\section*{Results}





We first conducted simulations to find points in axis space 
representing stable equilibria.
For these simulations, we used two axes that were both
neutral (i.e., $\mathbf{d} = \{0,0\}$) and thus had
no direct effect on other species.
This allowed us to determine where species evolve when investments
made by other species have no effect.
In this case, the number of species in the community and 
tradeoffs associated with investing in multiple axes
determine the location(s) of these stable axis states
(Figure \ref{fig:two-axis-outcomes}).
Species invest more in competition axes when there are
more species in the community, and make no investments
when they are the only species present.
Sub-additive tradeoffs 
(when investing in one axis facilitates investment in another; $\eta < 0$), 
result in a single stable point where both axes are
maximized---within the limits imposed by axes' negative effects on 
the growth rate.
When the tradeoffs are super-additive ($\eta > 0$), there are two
alternative stable states, one for each axis being maximized while the 
other is zero.
Lastly, for additive tradeoffs ($\eta = 0$), axes
evolve to any point on a neutrally stable ring.


% The locations of these states are determined by 
% the baseline growth rate, 
% the cost of increasing axes on the growth rate,
% and the non-additive tradeoff
% (Equations \ref{eq:two-axes-finals-eta-negative}--%,
% % \ref{eq:two-axes-finals-eta-zero}, and 
% \ref{eq:two-axes-finals-eta-positive}).


The next set of simulations explored how axis strength and axis investments
within communities affected coexistence.
We simulated 2-species, 2-axis communities where one axis was conflicting
and the other ameliorative.
We found all possible stable communities for scenarios with varying 
strengths of the two axes for both sub- and super-additive tradeoffs.
For each community and scenario, we calculated the 
scaled community size
($\sum_j^n{N_j \text{exp} \{ -\mathbf{v}_{j}^{\text{T}} \mathbf{D} \mathbf{v}_j \}}$), which represents the community abundance accounting
for the effect of species' axes on competition experienced by others in the
community.
A higher value of the scaled community size indicates a lower invasibility.
However, this trend only occurs within the same type of tradeoffs
because the costs associated with different tradeoffs are not the
same when summed across axis space (Figure \ref{fig:tradeoffs-r}).
From these simulations, we find that community type and axis strength
have strong effects on coexistence, as does their interaction
(Figure \ref{fig:comms-d}).
Super-additive tradeoffs always allow for more possible communities,
with up to 3 communities across much of the range of axis strengths.
Sub-additive tradeoffs only ever allow for 1 possible community,
but axis strengths can cause a switch in the possible community
(communities v and vi in Figure \ref{fig:comms-d}a).
Sub-additive communities are always sensitive to axis strengths
because all species always invest in both axes.
However, in super-additive communities, each species invests in 
only one axis, and this results in large variation in how
super-additive communities are affected by axis strengths.
For example, when both species invest in the conflicting axis,
that community's invasibility is greatly reduced with a
stronger conflicting axis (e.g., changing conflicting axis for community i).
Alternatively, strength of an axis that neither species invests in has
no effect on invasibility (e.g., community i with changing ameliorative axis).

We also find that increasing axis strength does not necessarily produce
outcomes consistent with how that axis affects coexistence
(greater coexistence with ameliorative, less with conflicting).
This is because varying axis strengths can change which
communities are stable.
For sub-additive tradeoffs, we find that increasing the ameliorative
axis strength past a certain threshold causes a new community to 
be the lone stable possible community (v changes to vi in Figure \ref{fig:comms-d}).
This new community has a greater competitive effect on invaders
(Figure \ref{fig:comms-d}a) and is more susceptible to losing a resident
species when an invader invests in the conflicting axis
(Figure \ref{fig:comms-d}b).
For super-additive tradeoffs, a community where both species invest in
the conflicting axis is no longer stable with a strong conflicting
axis, nor is a community with both investing in the ameliorative axis
stable for a strong ameliorative axis
(i and iv, respectively, in Figure \ref{fig:comms-d}).

Because the investments made by community members appears to be so important,
we next simulated equilibrium, 10-species communities with super-additive 
tradeoffs where 0--10 species invested in the conflicting axis
(the others invested in the ameliorative axis).
More species investing in the conflicting axis increases the investment
per species for all species (Figure \ref{fig:stabilizers}a,b).
A stronger conflicting axis generally causes a greater investment per species,
but this effect levels off with greater numbers of species investing in 
the conflicting axis.
The exception to this is when only one species invests in the conflicting axis.
In this case, the investment per species in the conflicting axis is lower
when then conflicting axis is strong.
This is because when the conflicting axis is strong and there is only one
species that invests in it, that species strongly suppresses all other species 
without experiencing competition from others investing in the conflicting axis.
The result is a lower scaled community size experienced by the species
investing in the conflicting axis, and species invest less with greater
scaled community size (Equation \ref{eq:2-axis-super-solns}).
The number of species investing in the conflicting axis and 
conflicting axis strength both also increased the scaled community 
size, an effect that levelled off more quickly for a strong conflicting axis
(Figure \ref{fig:stabilizers}c);
this levelling off was not as apparent when costs to investment ($f$) were reduced.
The effect of ameliorative axis strength was opposite to that of conflicting axis
strength (Figure \ref{fig:stabilizers2}).
Additionally, the effects of number of species investing in the conflicting
axis was much less asymptotic with a strong ameliorative axis than for a strong
conflicting axis.

Analytical results indicate that adding stochasticity to axis evolution
changes expected values of axes similar to changes in tradeoffs
(Equations \ref{eq:taylor-expansion-final}, \ref{eq:taylor-expansion-final-supp}).
Simulations when tradeoffs are weak ($| \eta |= 0.01$) support this.
The first example shown is when tradeoffs are super-additive and 
variance in axis-evolution stochasticity is the same across both axes
(Figure \ref{fig:stochasticity}a).
When variance is low, the simulations act similarly to when there is no
stochasticity, and species evolve to maximize one axis while the other is zero.
With a higher variance, a third attracting fixed point forms where species
invest in both axes.
With even higher variance, the first two points are no longer attractors,
resulting in species evolving to a single point where both axes are maximized.
Axis evolution thus behaves similarly to when tradeoffs are sub-additive.
Increasing variance in this case causes a decrease in scaled community size,
but changes in species starting axis values can result in the 
opposite effect (Figure \ref{fig:alt-stochasticity}).
Outcomes with sub-additive tradeoffs are not similarly affected when
stochasticity variance is equal for both axes (Figure \ref{fig:stochasticity}b).
However, if the variance in axis-evolution stochasticity is greater in one axis,
the attracting fixed point moves such that it is reduced for that axis
and increased for the other.
Therefore, a higher variance in the conflicting axis makes the community
more invasible, while higher variance in the ameliorative axis makes it less.

