\section*{Results}





We first conducted simulations to see how investment-cost additivity 
determines the locations in investment space to which species evolve
 (Figure \ref{fig:tradeoffs-outcomes}).
We find that sub-additive costs (when investing in one suite facilitates
investment in the other; $\eta < 0$) result in a single outcome where all 
species invest in both suites of traits.
Additive costs ($\eta = 0$) result in a neutrally stable ring, where
any location along that ring is an equally viable investment strategy.
Lastly, when investment costs are super-additive ($\eta > 0$), there are two
alternative stable states, one where species invest only in conflicting traits,
another where they only invest in partitioning traits.


The relative investment in each suite of traits is also affected by
non-adaptive stochasticity, especially when investment costs are weakly
non-additive (Figure \ref{fig:stochasticity}).
When stochastic variances are equal for both suites of traits, 
this has no effect when costs are sub-additive (Figure
\ref{fig:stochasticity}\textit{A}), but
for super-additive costs, increasing variances results in a third
attractor point where species invest in both suites of traits
(Figure \ref{fig:stochasticity}\textit{B}).
Further increasing variances widens the basin of attraction for this point,
potentially resulting in most species evolving to invest in both suites 
of traits.
When variances are unequal, this reduces the investment for the suite of traits 
with the greater variance.
When costs are sub-additive, the single attractor point in investment
space moves away from the axis for the suite of traits with greater variance 
(Figure \ref{fig:stochasticity}\textit{A}).
When costs are super-additive, the basin of attraction widens for the point
where species only invest in the suite of traits with lower stochastic variance
(Figure \ref{fig:stochasticity}\textit{B}).
If the distribution of starting investments are symmetrical across investment
space, this should result in more species specializing in the suite of traits
with weaker non-adaptive stochasticity.
That non-adaptive stochasticity affects investment evolution is also 
supported by analytical results (Equation \ref{eq:taylor-expansion-final}).


The total amount invested unsurprisingly depends on the investment costs
(which are fixed in this model) and benefits.
The benefits depend on the number, abundances, and investments of other 
species in the community.
If there are many other species in the community that are all abundant 
and invested heavily in conflicting traits, this will create selection 
for the focal species to invest more heavily to avoid this strong 
interspecific competition.
On the other extreme, where there are few others competitors that are rare
and invested heavily in partitioning traits, selection would favor lower
investments since the cost of investing is less justified.
In our model, where a community sits on this gradient, from the perspective
of species $i$, can be calculated as
$\sum_{j \ne i}^{n}{ N_j \text{e}^{ d_x x_j^2 - d_p p_j^2 } }$.
We call this the ``scaled community size'' ($\Omega_i$) and define it as the
total community abundance accounting for the effect of species' investments 
on competition experienced by species $i$.
Our analytical solutions show that when there are no competitors, 
species should not invest at all
(Equation \ref{eq:analytical-1spp-solns})
and that in multi-species communities, 
total investment is proportional to $\Omega_i$
(Equations \ref{eq:analytical-sub-solns}, \ref{eq:analytical-super-solns}).



The effect of scaled community size on total investment results in feedbacks
between community investment composition and subsequent investment by new
species (Figure \ref{fig:community-invasions-zero}).
When all members of a community invest in partitioning traits, this reduces
the total amount invested per species for a given community size, with the
potential for some species to avoid investment altogether.
In Figure \ref{fig:community-invasions-zero}\textit{A--C}, all species invading
the community invest only in partitioning traits. 
The species that first invades (species 1) evolves to invest less during 
the period it remains alone, so when the others arrive, it has the lowest 
total investment in the community.
Because of these starting conditions and because a community full of
partitioning investment selects for low investment,
species 1 slowly evolves to not invest at all.
Figure \ref{fig:community-invasions-zero}\textit{D--F} shows that this can also 
happen when one of the species (species 3) invests only in conflicting traits.
The main differences here are that
(1)~the species evolving to no investment does not need to start with 
less overall investment than the other species and
(2)~the evolution to zero investment occurs faster.
Moreover, before species 3 evolves to invest less, it causes a precipitous
decline in the abundances of the other species and a slightly less 
dramatic increase in their total investments.
Despite species 1 being more abundant, it declines more severely than 
species 2  after species 3 arrives.
This is because species 1 is less invested in reducing interspecific competition
and thus more vulnerable to a species investing in conflicting traits.


Figure \ref{fig:community-invasions-armsrace}\textit{A--C} shows that
the effects of strong partitioning investment in a community can be quickly
torn apart when just two species arrive that invest in conflicting traits.
This is because these two species engage in an arms race where one invests
more heavily and expands in population, then slowly invests less as others
in the community become less abundant.
The other species then evolves to invest more in conflicting traits
because of the increased competition it experiences from the abundant species,
and it subsequently becomes most common.
This results in damped oscillations that eventually reach an equilibrium 
where both species have invested much more than when they started.
During this back-and-forth, the species that invest only in partitioning 
traits increase their own investments to compensate for the increasing 
competition they experience.
Invaders investing in partitioning traits are excluded unless
they have already invested heavily.
Whether this arms race occurs depends on the time between successive 
invasions of species investing in conflicting traits
(Figure \ref{fig:community-invasions-armsrace}\textit{D--F}).
With a large gap between conflicting investors,
the early-arriving species, with only partitioning investors to compete with,
evolves a low enough investment that it is excluded when a new 
conflicting investor arrives.
The partitioning investors that had already evolved increased investment
in response to each other and the first conflicting investor
survive both invasions of conflicting investors.
The partitioning investor that evolved no investment quickly went extinct
upon invasion of the first species with conflicting investment.
A new invader with moderate investment in partitioning traits
can coexist in this community because of the reduced investment
evolved by the conflicting investor.
These results are similar to those where resident species invest in
only conflicting traits:
If the gap between new invaders is short, the invader must have
invested heavily in either suite of traits to successfully invade.
If the gap is long, then invaders more heavily invested in conflicting traits
than the resident will exclude the resident that has evolved reduced investment
while it is alone;
moderately invested partitioning invaders will coexist with the resident.


In the previous two examples (Figures \ref{fig:community-invasions-zero} and
\ref{fig:community-invasions-armsrace}), costs are super-additive, and 
to adjust the community-wide strength of a given suite of traits, 
we simply change the proportion of species specializing in it.
Communities with sub-additive costs can similarly differ in the effects of
existing investment in partitioning versus conflicting traits.
This occurs not because of differences in investments among species since 
they all eventually invest in both suites equally.
Instead, this happens when one suite of traits has a stronger effect on other
species for each unit invested (i.e., $d_x \ne d_p$).
Whichever suite has a greater per-unit-invested effect is more influential 
as species evolve to invest in both, and new species' total investments are
affected in a qualitatively similar way to when tradeoffs are super-additive
(Equations \ref{eq:analytical-sub-solns}, \ref{eq:analytical-super-solns}).
The exception to this is when $\eta \le -1$, a scenario where investing
in both suites of traits in the same amounts will cost nothing or even 
provide a benefit to the per-capita growth rate.
This lack of tradeoff results in ever-increasing investments that are
unlikely in natural populations.
A more important difference between the sub- and super-additive cases
is that the feedbacks between current community investments and those
of new invaders are usually transient when costs are sub-additive because
the invaders always evolve to invest in both suites of traits.
The only time these feedbacks have lasting effects is when invaders
evolve to not invest or instigate a resident species to do the same, 
which can also occur when $d_x \ll d_p$.
Exactly the same dynamics as the super-additive case could be observed
if species differed in their values of $d_x$ and $d_p$, which 
we do not include our model but would of course be the case in real communities.
In summary, we focused on the super-additive case because
(1)~the feedbacks are qualitatively the same,
(2)~any differences in dynamics are caused by unrealistic assumptions
of the model, and
(3)~the super-additive case is easier to visualize.






