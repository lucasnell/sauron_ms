\section*{Results}


Our model of ecological and evolutionary dynamics investigates selection
for partitioning and conflict traits, and how selection structures
competition in the resulting community. To illustrate the central
differences between partitioning and conflict traits, consider the case
of two species and the effects on species B when species A changes its
investment in either a partitioning or a conflict trait (fig. \ref{fig:model-description}). To
clarify this illustration, we fix the abundance and investment
level of species A, rather than allow them to change as described above.
When species A invests in a partitioning trait, the abundance of species
B increases (fig. \ref{fig:model-description}B), because greater investment in partitioning by
species A reduces competition experienced by species B. With reduced
competition, selection favors lower investment by species B. In
contrast, for a conflict trait increased investment by species A causes
a decrease in abundance of species B, and species B experiences
selection for increased investment in the conflict trait (fig. \ref{fig:model-description}C).
Thus, investment in partitioning and conflict traits by species A have
opposite effects on the abundance and selection for species B.

\subsection*{Investment-cost additivity}

We first show how different types of communities arise due to sub- and
super-additive investment costs (fig. \ref{fig:tradeoffs-outcomes}). 
We simulated 2-species communities
where investment costs were either sub-additive ($\eta = - 0.6$),
additive ($\eta = 0$), or super-additive ($\eta = 0.6$). We added
both species at the same time to avoid effects of evolution before
interspecific competition starts. We set both suites of traits to be
nearly neutral ($d_{x} = d_{p} = 10^{- 6}$) to reduce the effects of 
species’ investments on each other, which helps to highlight the effects of 
cost additivity only.
The starting investments
by all species were generated from a uniform distribution between 0 and
0.5. Sub-additive costs resulted in species evolving to a single point
in investment space where each invests in both conflict and partitioning 
traits equally. Super-additive costs resulted in two
alternative stable investment states: species evolved to invest in only
partitioning or only conflict traits depending on their investments at
the start of the simulations; different species could occupy different
investment states even within the same community.
Additive costs resulted in an intermediate
case, with a neutrally stable ring given by
$\sqrt{{p_{i}}^{2} + {x_{i}}^{2}}$ containing an infinite number of
investment strategies with the same fitness; like the super-additive case,
species within the same community can occupy different states.
Because the total investment ($\sqrt{{p_{i}}^{2} + {x_{i}}^{2}}$) 
defining a ring is proportional to both the costs (eq. \ref{eq:growth-rate})
and benefits (eq. \ref{eq:competition}) of investment, the ring observed in
the additive case is also present in the other two cases.
For sub-additive and super-additive costs, species trait values evolve
towards the $\sqrt{{p_{i}}^{2} + {x_{i}}^{2}}$ ring and then either
converge to equal investment in partitioning and conflict traits
(sub-additive) or diverge to investment in only partitioning or only
conflict traits (super-additive) (fig. S1).
Furthermore, larger magnitudes of $\eta$ cause more rapid evolution around
the ring (fig. S2).
These results can be derived analytically
(eqs. \ref{eq:analytical-super-solns}--\ref{eq:analytical-sub-solns}).


Sub-additivity and super-additivity generate communities that differ in
structure. At the community scale, sub-additive investment costs allow
for only one possible stable configuration, where all species invest in
both suites. For super-additive costs, there are three stable 2-species
configurations: both species investing in partitioning traits, both
investing in conflict traits, and one species investing in each. There
are infinite community configurations for additive costs.

\subsection*{Non-heritable variation}

Because non-heritable variation can alter the effects of selection, we next
explored the effects of non-heritable variation using simulations in
which communities contained four species each with an initial total
investment $\sqrt{{p_{i}}^{2} + {x_{i}}^{2}}$ = 1. Two species had
initial investments favoring partitioning traits, and the other two had
higher initial investments in conflict traits (fig. \ref{fig:non-heritable}). We considered
three levels of non-heritable variation: non-existent
($\sigma_{p} = \ \sigma_{x} = 0$), low
($\sigma_{p} = \ \sigma_{x} = \ 0.11$), or high
($\sigma_{p} = \ \sigma_{x} = 0.2$). We crossed these levels to give
nine permutations of non-heritable variation, for example,
$\sigma_{p}\  = \ 0.2$ and $\sigma_{x}$ = 0 in the top-left panels
of figs. \ref{fig:non-heritable}A, \ref{fig:non-heritable}B, and \ref{fig:non-heritable}C (where the non-heritable variation is given in the
gray distributions on the sides of the panels). For the case when
non-heritable variation affects partitioning and conflict traits the same
(figs. \ref{fig:non-heritable}A, \ref{fig:non-heritable}B, \ref{fig:non-heritable}C, subpanels along the diagonal),  adding non-heritable variation causes the additive and super-additive cases to behave like the sub-additive case, whereby the neutrally stable ring and alternative stable states are replaced by a single stable state with investment in both partitioning and conflict traits. Furthermore, non-heritable variation reduces the efficacy of selection for a given trait; for example, when non-heritable variation for the partitioning trait is greater than for the conflict trait, the stable investment state has higher investment in the conflict trait. This is a manifestation of the classic result in quantitative genetics whereby the efficacy of selection is scaled by the additive-genetic variance (i.e., heritable) relative to the non-heritable variance 
\citep{Fisher1930a, Wright1931b, Barton2017}.
These results are derived analytically
in Appendix A (eq. \ref{eq:taylor-expansion-final}).
However, our results for non-heritable variation only apply for 
weakly super-additive traits. When costs are
strongly non-additive ($\eta = - 0.6$ or $\eta = 0.6$), non-heritable
variation has little effect (fig. S3).

\subsection*{Coevolution when communities are invaded}

Non-additivity of costs ($\eta$) and non-heritable variation
($\sigma_{p}$ and $\sigma_{x}$) affect the possible states to which
species can evolve in 2-species communities (figs. \ref{fig:tradeoffs-outcomes}
and \ref{fig:non-heritable}). To address
more broadly how competitive trait coevolution affects the structure of
communities, we consider the invasion of 2-species communities by a
competitor that invests in either partitioning or conflict traits, and
analyze the ecological (abundance) and evolutionary (trait investment)
responses of the resident species and the invading species. Our focus is
on the final configuration of the community, specifically the investment
of species in partitioning and conflict traits, and whether this
configuration is stable to the addition of new species exhibiting
different investments.

A key factor needed to understand the evolution of trait investment for
species is the total strength of competition it experiences, which
depends on the number, abundances, and trait investments of the other
species in the community. For example, if there are many other species
in the community that have high population abundance and invest heavily
in conflict traits, this will create strong selection for the focal
species to invest in traits to decrease interspecific competition. The
strength of interspecific competition experienced by species $i$ is
given by
$\sum_{j \neq i}^{n}{N_{j}\text{e}^{- d_{p}p_{j}^{2} + d_{x}x_{j}^{2}}}$.
We call this the ``effective competitive neighborhood'', $\Omega_{i}$.
Analytical solutions show that the optimal investment for a species is
proportional to the interspecific competition it experiences at the
stationary state, denoted ${\hat{\Omega}}_{i}$
(eqs. \ref{eq:analytical-1spp-solns}--\ref{eq:analytical-sub-solns},
fig. S4).

To investigate the effect of trait investments by both resident and
invading species on community coevolution, we conducted simulations
where we varied whether resident communities and the invader invest in
partitioning or conflict traits in a 2x2 design. We assumed additive
costs ($\eta = 0$) for residents and invaders. Because costs were
additive, we could set whether residents invested more in partitioning
or conflict traits by selecting appropriate trait starting values in the
simulations. Starting with two resident species with greater investment 
in either partitioning or conflict traits, we first ran the simulations 
to stationarity for the residents. We then introduced the invader
that either invested mostly in partitioning traits or in conflict
traits. To allow the direct comparison between partitioning and conflict
traits, we assumed that they have the same magnitude of effects on the
non-focal species ($d_{p} = d_{x} = 0.6$). 
Because we were interested in coevolutionary patterns and not exclusion, we 
also chose values of $d_p$ and $d_x$ that allowed both residents and invaders 
to persist.

When both residents and the invader invested mostly in conflict traits
(fig. \ref{fig:one-invader}A), the residents experienced greater effective competitive
neighborhoods, which selected for an increase in their overall
investment and caused a decrease in their abundance. The invader also
evolved greater overall investment and had an equilibrium abundance
similar to the residents. The effective competitive neighborhood for the
invader changed little because the decrease in the abundance of
residents offset the effect of their increased investments. When a
species investing in partitioning traits invaded the same resident
community (fig. \ref{fig:one-invader}B), investments and abundances of residents changed
little. The invader, however, evolved greater investment and reached a
lower abundance than the residents. The abundance and investment by the
partitioning invader matched those of the conflicting invader when they
invaded the same community (fig. \ref{fig:one-invader}A,B). When a species investing mostly
in conflict traits invaded a community of partitioning residents, this
increased the investment and decreased the abundance for residents, and
resulted in low investment and highest abundance for the invader (fig.
\ref{fig:one-invader}C). An invader that invested in partitioning traits invading a resident
community of species investing in partitioning traits resulted in little
change for all species (fig. \ref{fig:one-invader}D).

Overall, these simulations show that investment in conflict traits by
both invader and resident species leads to additional investment,
whereas investment in partitioning traits leads to relatively less
selection for changes in investment. This is true for the effects of
invaders on residents (fig. \ref{fig:one-invader}A,C versus \ref{fig:one-invader}B,D) and of residents on
invaders (fig. \ref{fig:one-invader}A,B versus \ref{fig:one-invader}C,D). However, these contrasting effects of
investing in conflict versus partitioning traits are not necessarily
symmetrical.

When species $i$ invests in either type of trait, it reduces its own
competition, which allows it to increase in abundance. An increase in
$N_{i}$ increases competition for all other species in the community
(i.e., it increases $\Omega_{j}$ for all $j\  \neq i$). If species
$i$ had invested in partitioning traits, this increase in $N_{i}$
undercuts the competition-reducing effects of increasing $p_{i}$ on
all other species. If species $i$ had invested in conflict traits, the
increase in $N_{i}$ exacerbates the competition-increasing effects of
increasing $x_{i}$. Therefore, conflict trait investment by one
species increases the competition experienced by other species through
its effects on both the investor's abundance and traits, whereas when a
species invests in partitioning traits, the abundance- and
trait-mediated effects on other species counteract one another. This can
be seen in the simulations when residents invested heavily in conflict
traits and invaders in partitioning traits (fig. \ref{fig:one-invader}B) and when residents
invested heavily in partitioning traits and invaders in conflict traits
(fig. \ref{fig:one-invader}C), where we found that the species investing in conflict traits
had a greater influence on investments than those investing in
partitioning traits. Thus, in communities with species expressing both
partitioning and conflict traits, we expect the evolution of conflict
traits to be the stronger driver of trait evolution for all species.

\subsection*{Competitive exclusion}

In our model, species that invest more in conflict traits are generally more
abundant and invest less overall than species investing mostly in partitioning
traits (fig. \ref{fig:one-invader}). If an invader invested heavily in conflict
traits---the type most likely to cause competitive exclusion---arrives
to a community, greater total investments by residents help shield them from 
the new source of competition, whereas greater resident abundance should 
slow invader population growth and provide the resident a longer time for
evolutionary rescue via greater investment. Our next simulations assess how
these factors together affect the likelihood of an invader excluding resident
species that invest in either partitioning or conflict traits.

We simulated the invasion of 2-species equilibrium communities similar
as in the last section, but with one resident species always invested in
conflict traits and the other always invested in partitioning traits.
The invader only invested in conflict traits (invader
$p_{t = 0} = 0,\ x_{t = 0} = 1$).


We simulated the invasion of 2-species equilibrium communities as in the
previous section, but with one resident species always invested more in
partitioning traits (partitioning resident: $p_t \gg x_t$) and the other 
always invested more in conflict traits (conflict resident: $p_t \ll x_t$). 
We varied the strength of effects of
trait investment on other species ($d_{p}$ and $d_{x}$) because
these parameters shape interactions between residents and invaders.
We varied the strength of the effect of investing in traits on other species
($d_{p}$ and $d_{x}$) because these parameters shape interactions between
residents and invaders, but we focused on $d_p$ since this had the strongest
effect on the coevolutionary dynamics (fig. S5). 
At equilibrium, the conflict resident invested less overall and was more
abundant than the partitioning resident, and these disparities increased with
$d_p$ (fig. \ref{fig:exclusion}A,B). Specifically, when $d_p$ = 0.3, the
conflict resident moderately invested, while it invested very minimally when
$d_p$ = 0.7; the portioning resident maintained high investment in both
scenarios (fig. \ref{fig:exclusion}C,D). When an invader investing in conflict
traits (conflict invader: $p_{t=0} = 0$, $x_{t=0} = 1$) was added to the
equilibrium 2-species community, overall investment was more important than
abundance in determining which resident was excluded.
When the conflict resident invested moderately ($d_p=0.3$;
fig. \ref{fig:exclusion}C) at the 2-species equilibrium, the addition of the
invader caused only a temporary drop in the conflict resident’s abundance in
response to the increase in the effective competitive neighborhood (fig.
\ref{fig:exclusion}E). In contrast, when the conflict resident invested
minimally ($d_p=0.7$; fig. \ref{fig:exclusion}D), it went extinct as it was
unable to increase its investment rapidly enough in response to the increase in
the effective competitive neighborhood (fig. \ref{fig:exclusion}F).
Because the partitioning resident always
invested more overall, the effects of the conflict invader were less
severe and did not cause it to go extinct. This pattern where the
conflict resident was easier to exclude than the partitioning resident
was constant over a wide range of $d_{p}$, $d_{x}$, and the conflict
invader's starting total investment (fig. S6).


These simulations show that investing in conflict traits can make
species more likely to be excluded from a community that also contains
species that invest in partitioning traits. By reducing other
competitors' abundances, thereby selecting for reduced investment for
itself and increased investment for others, competitors invested
proportionally more in conflict traits make themselves the most vulnerable to
exclusion despite being the most abundant in the community. We might
therefore expect higher turnover and greater variation in abundance for
species invested in conflict traits. This bias in exclusion risk could
result in communities that, on average, contain fewer species investing
strongly in conflict traits.

