
\section*{Methods}


\subsection*{Model overview}

We used a simple model of competition to evaluate how competition affects
outcomes of community assembly and trait evolution.
Each of $n$ species has $q$ traits that affect competition.
All traits and species are symmetrical.
Increases in traits benefit the species via reduced
effects of both interspecific and intraspecific competition;
they also incur a cost by decreasing the species' population growth rate.
In both the costs and benefits, trait effects are concave functions that,
combined, ensure that one fitness peak exists for each trait.
Traits can also have non-additive trade-offs that either increase or decrease
the cost associated with increasing multiple traits compared to increasing
just one.
Trait evolution in one species also affects competition experienced by others.
This can be either conflicting or non-conflicting:
When conflicting, increasing traits in one species increase the effects of
competition for all other species.
When non-conflicting, all competitors' competition is reduced when traits
increase in one species.

% Below, we use matrix notation that applies to any number of traits.
% Matrices (indicated by bold face) are multiplied using matrix---not
% element-wise---multiplication,
% and ${}^{\textrm{T}}$ represents matrix transposition.
We used a discrete-time, modified Lotka--Volterra competition model similar to
that by \citet{Northfield:2013if}.
In it, species $i$ has a $1 \times q$ matrix of traits ($\mathbf{V}_i$), and
its per-capita growth---equivalent to fitness---is

\begin{equation} \label{eq:fitness}
    F_{i} = \exp \left\{ r_i(\mathbf{V}_i) - \alpha_{ii}(\mathbf{V}_i) N_i - \sum_{j \ne i}^{n}{
        \alpha_{ij}(\mathbf{V}_i, \mathbf{V}_j) N_j}  \right\}\textrm{,}
\end{equation}

\noindent where $N_i$ is the population density of  species $i$.
The parameter $r(\mathbf{V}_i)$ describes how a trait increase
decreases the growth rate:

\begin{equation} \label{eq:growth-rate}
\begin{split}
    r(\mathbf{V}_i) &= r_0 - f ~ \mathbf{V}_i ~ \mathbf{C} ~ \mathbf{V}_{i}^{\textrm{T}} \\
    \mathbf{C} &= \begin{pmatrix}
        1         & \ldots & \eta_{1q} \\
        \vdots    & \ddots & \vdots \\
        \eta_{q1} & \ldots & 1      \\
        \end{pmatrix}
    \textrm{,}
\end{split}
\end{equation}

\noindent where $r_0$ is the baseline growth rate,
$f$ is the cost of increasing traits on the growth rate, and
$\eta_{k,l}$ is the non-additive trade-off of increasing both the
$k$\textsuperscript{th} and $l$\textsuperscript{th} traits.
When $\eta > 0$, increasing multiple traits incurs an extra cost.
Non-additive trade-offs are symmetrical (i.e., $\eta_{k,l} = \eta_{l,k}$ for all
$l$ and $k$), and all values on the diagonal of $\mathbf{C}$ are 1.
The relationships between trait values and other components of fitness
(growth rates and effects of competition) are of the form
$v^2 \propto X$ for parameter $X$, so $v = z$ is equivalent to $v = -z$.
To avoid alternative outcomes due to artifacts of this relationship,
traits are not allowed to be $< 0$.
We did this by passing the equation for $\mathbf{V}_{t+1}$ through a
ramp function.
This implementation and its consequences on resulting derivatives are in
Appendix A.


The terms $\alpha_{ii}(\mathbf{V}_i)$ and
$\alpha_{ij}(\mathbf{V}_i, \mathbf{V}_j)$
in equation \ref{eq:fitness} represent how a trait decreases the effects
of intraspecific and interspecific competition, respectively.
Competition effects are given by

\begin{equation} \label{eq:competition}
\begin{split}
    \alpha_{ii}(\mathbf{V}_i) &= \alpha_0 ~\exp \left\{- \mathbf{V}_i
        \mathbf{V}_i^{\textrm{T}} \right\} \\
    \alpha_{ij}(\mathbf{V}_i, \mathbf{V}_j) &= \alpha_0 ~\exp \left\{
        - \mathbf{V}_i \mathbf{V}_i^{\textrm{T}} -
        \mathbf{V}_j \mathbf{D} \mathbf{V}_j^{\textrm{T}} \right\} \\
    \mathbf{D} &= \begin{pmatrix}
        d_1     & \ldots    & 0 \\
        \vdots  & \ddots    & \vdots \\
        0       & \ldots    & d_q
        \end{pmatrix}
	\textrm{,}
\end{split}
\end{equation}

\noindent where $\alpha_0$ is the base density dependence.
Matrix $\mathbf{D}$ contains parameters that determine how evolution of traits
in one species affects competition experienced by others:
When $d_k < 0$, increases in $v_{i,k}$ decrease the
effect of competition on species $i$, but increase it in all others.
Alternatively, when $d_k > 0$, increases in $v_{i,k}$ decrease the effect of
competition on all species including itself.
Thus $d_k < 0$ causes conflicting and $d_k > 0$ causes non-conflicting evolution
for trait $k$ \citep{Northfield:2013if}.
We could expect conflicting evolution to occur when contest competition
leads to arms races among competitors
\citep{Abrams:1994th}.
Competitor evolution might be non-conflicting when they evolve
dissimilar traits to reduce competition for resources \citep{Roughgarden:1976eh}.


If we combine the above equations and simplify them, we get

\begin{equation} \label{eq:fitness-full}
\begin{split}
    F_{i} &= \exp \left\{
        r_0 - f ~ \mathbf{V}_i ~ \mathbf{C} ~ \mathbf{V}_{i}^{\textrm{T}} -
        \alpha_0 ~\textrm{e}^{- \mathbf{V}_i \mathbf{V}_i^{\textrm{T}} } \mathbf{\Omega}_{i}
        \right\} \\
        \mathbf{\Omega}_i &\equiv N_i +
            \sum_{j \ne i}^{n}{ N_j \textrm{e}^{ - \mathbf{V}_j \mathbf{D} \mathbf{V}_j^{\textrm{T}} } }
        \textrm{,}
\end{split}
\end{equation}

\noindent where $\mathbf{\Omega}_i$ represents the community abundance scaled
for the effect other species' trait values has on competition
experienced by species $i$.
Hereafter we will refer to $\mathbf{\Omega}_i$ as the scaled community size.




% \subsection*{Adaptive dynamics}
%
% We started simulations with a single competitive species with trait values set to zero.
% We tracked species population densities through time using equation \ref{eq:fitness} and
% considered a species extinct if its density fell below $10^{-4}$.
% Species produced daughter species with a probability of 0.01 per species per time step.
% We generated daughter-species trait values from normal distributions with means of the
% mother trait values and standard deviations of $\sigma_{d}$.


\subsection*{Quantitative genetics}

We used a quantitative genetics framework for trait evolution.
We assumed that all traits in $\mathbf{V}_i$ represent means for species $i$
and that their among-individual distributions are symmetrical with additive
genetic variance $\sigma^2_i$.
Assuming also that $\sigma^2_i$ is relatively small
\citep{Iwasa:1991eo,Abrams:2001va,Abrams:1993cr}, traits at time $t+1$ are

\begin{equation} \label{eq:trait-change}
    \mathbf{V}_{i,t+1} = \mathbf{V}_{i,t} + \left( \frac{1}{F_i}
        \frac{\partial F_i}{\partial \mathbf{V}_{i}} \right) \sigma^2_i
    \textrm{.}
\end{equation}

To determine the stability of ending points (trait values and abundances of
surviving competitor(s)), we computed the $nq \times nq$ Jacobian matrices
of first derivatives for the traits of each species:

\begin{equation} \label{eq:jacobian}
    \begin{pmatrix}
        \frac{\partial \mathbf{V}_{1,t+1}}{\partial \mathbf{V}_{1,t}} & \cdots &
            \frac{\partial \mathbf{V}_{1,t+1}}{\partial \mathbf{V}_{n,t}} \\
        \vdots & \ddots & \vdots \\
        \frac{\partial \mathbf{V}_{n,t+1}}{\partial \mathbf{V}_{1,t}} & \cdots &
            \frac{\partial \mathbf{V}_{n,t+1}}{\partial \mathbf{V}_{n,t}}
    \end{pmatrix}
    \textrm{.}
\end{equation}

\noindent We then computed the primary eigenvalue of this matrix ($\lambda$).
We considered a state stable when $\lambda < 1$,
neutrally stable when $\lambda = 1$,
and unstable when $\lambda > 1$.

Full analytical solutions to matrix derivatives (for trait change and
Jacobian matrices) are found in appendix A.

We also analyzed equilibrium solutions for the 2 trait case.
This is found in Appendix B.





\subsection*{Simulations}

We started simulations with 100 species, each of which
had starting trait values generated from a truncated normal distribution 
with a standard deviation of 2 and a lower bound of zero.
Species started with an abundance of 1.
We tracked densities through time and considered a species extinct if its 
density fell below $10^{-4}$.
Thus, our simulations resemble those of traditional 'filtering' models
in community ecology, 
except that species' traits are not fixed through time.

When generating multiple values of $\eta$ to simulate different 
types of tradeoffs, 
we did so using a uniform distribution from 0.1 to 0.4.
We intentionally avoided the use of simple $\eta$ values because
when the sum of one or more $\eta$ equals another, this simplifies
to additivity even when those $\eta \ne 0$.
This only applies in the case of $q > 2$, since there is only one
$\eta$ value when $q = 2$.






\subsection*{Code}

We simulated models using a combination of R \citep{RCoreTeam:2019wf} and
C++ via the Rcpp and RcppArmadillo packages
\citep{Eddelbuettel:2014ad,Eddelbuettel:2013if,Sanderson:2016cs}.
We double-checked our derivations by simulating 100 datasets
(4 species with 3 traits each) and computing derivatives using the Theano Python
library's automatic differentiation \citep{TheanoDevelopmentTeam:2016uc}.
All code can be found on GitHub
% AmNat does double-blind reviews, so the next line will be un-commented after review
% (\texttt{https://github.com/lucasnell/sauron}).
(links are available from the journal office).

