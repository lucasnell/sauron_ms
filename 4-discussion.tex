
\section*{Discussion}

% Here, we first showed how tradeoffs, initial conditions, and conflicting/ameliorative coevolution
% combine to set the relevant deterministic rules for coexistence in a competitive system.
% Tradeoff additivity (i.e., how the cost of investing in multiple axes together 
% compares to the sum of investing separately) shapes how fitness maps onto the axis space,
% creating stable fitness peak(s) or a neutrally stable ridge.
% For multiple peaks or a neutrally stable ridge, initial conditions (i.e., where in the 
% axis space species start) determine the peak or ridge location to which each species will evolve.
% The extent to which coevolution is conflicting (i.e., investing by one species increases 
% competition experience by all others) or ameliorative for each axis affects
% how permissible coexistence should be across the axis space.
% Thus, tradeoffs and initial conditions dictate where in the axis space evolution will take species, and 
% conflicting/ameliorative coevolution determines how well they can 
% coexist when they get there.





% We also showed how stochasticity could bend and break these deterministic rules.
% Stochasticity always affected coexistence via transient effects during the 
% critical period immediately after a new invader arrived to the community.
% Stochasticity in population dynamics had both positive and negative effects on
% coexistence, via varying effects on both resident and invading species.
% It increased the chances of rare species going extinct,
% but it could temporarily increase their abundance and help them persist.
% It could also increase or decrease fitness for
% the resident species at critical times for an invader, which
% could increase or decrease the chances for the invader to survive long enough 
% for evolutionary rescue.
% Stochasticity in evolution had both positive and negative effects on
% coexistence, but its positive effects were more consistent than for stochasticity in 
% population dynamics.
% It could reduce or increase the invader's ability for 
% evolutionary rescue by changing the distance in axis space from the phenotype to 
% to the equilibrium state, compared to the genotype.
% It also kept phenotypes from remaining at the optimal 
% axis values (i.e., $\ddot{v}_{ij} \ne \hat{v}_{ij}$),
% which affected resident species in two ways.
% First, it could increase or decrease their effects on other species in the community.
% Second, it reduced their fitness and abundance, which had a consistently 
% positive effect on coexistence.
% This latter effect is why evolution stochasticity had a more consistent positive effect
% on coexistence than population-dynamics stochasticity.
% Despite the different mechanisms, both types of stochasticity usually had qualitatively
% similar effects on coexistence:
% When deterministic rules allowed for coexistence among species, 
% stochasticity could only have negative effects on it.
% On the margins, just outside the bounds of where these rules permitted coexistence, 
% stochasticity improved the chances of multi-species coexistence.


% When tradeoffs were additive, stochasticity in axis evolution qualitatively changed
% the dynamics.
% Instead of evolving to any point along the neutral ring, species evolved to a specific
% location on that ring.
% The exact location of the ring depended on the variances of stochasticity among axes.
% The axis with a greater variance evolved to be lower.







% % A Conclusions section is common in AmNat, but not required
% % \section*{Conclusions}

