
\section*{Discussion}


We find that the costs for investing in both suites of traits dictate whether
species evolve to be specialists or generalists with regard to their evolved
mechanisms to reduce the interspecific competition they experience.
We might expect super-additive tradeoffs when investment in one suite of
traits makes it more difficult to achieve the same results for a given
investment in the other suite, such as in the case of body size and speed.
We might also expect this when both suites of traits require resources from a
limited pool.
In plants, growth--defense tradeoffs are thought to occur because plants 
have limited resources they can devote to growth and defense from herbivores, 
so investment in growth necessarily reduces their ability to invest in defense.
If investment in growth shades competitors and investment in defense produces
differences among species in their responses to natural enemies, this would
represent super-additive costs for investing in both conflicting and 
partitioning traits.
We would expect sub-additive costs when one suite of traits facilitates 
the other.
[[ ANY IDEAS ON EXAMPLES OF SUB-ADDITIVE COSTS? ]]



Feedbacks between investments in the community and future investment 
evolution occur because
(1)~the competition species experience is based on both abundances 
and investments by others in the community and
(2)~the investments each makes is proportional to the competition 
they experience.
A community with strong investment in partitioning traits will therefore 
select for lower investment.

This can result in a ceiling for the effects of niche partitioning:
If some competitors already invest in traits that are heavily reducing
competition, then selection may favor little or no investment in
partitioning traits for other species
(Figure \ref{fig:community-invasions-zero}A--C).
This occurred more easily in species-poor communities because the 
overall competition experienced was less.
Thus, in communities with strong niche partitioning,
especially those with relatively few species, we might
expect there to be some competitors that have evolved
little to no partitioning traits.



Heavy community-wide partitioning investment similarly reduces investment when a
conflicting investor invades (Figure \ref{fig:community-invasions-zero}D--F).
This means that partitioning investment can resist effects of conflicting
investment not just through the summed effect on the competition experienced 
by each species, but also by selecting for reduced investment by those
investing in conflicting traits.
Because conflicting investors have a much stronger effect on the abundances
of species investing in partitioning traits, reduced investment occurred
for even species-rich communities, assuming those communities only contained
partitioning investors.
Because partitioning investors cannot evolve to have a greater per-capita
competitive effect on conflicting investors than 
conflicting investors on partitioning investors, 
these results are similar to those by \citet{Kisdi2001}.
They found that asymmetric competition should often result in 
`evolutionary disarmament' where competitive weaponry should be reduced 
through time.



When two or more conflicting investors invade a strongly partitioning community,
the result is an arms race
(Figure \ref{fig:community-invasions-armsrace}A--C)
or serial exclusion of conflicting investors
(Figure \ref{fig:community-invasions-armsrace}D--F).
What separates these outcomes is the time between successive invasions
of conflicting investors:
When invaders arrive often, we would expect arms races between two
or more species that have invested heavily in conflicting traits
and exclusion of all partitioning investors not able to also evolve
strong investments.
When invaders arrive rarely, we would expect a rotating door of conflicting 
investors that continually get excluded from the community by invaders and
successful invasion and coexistence of partitioning investors.
The latter scenario results in a ratchet of coexistence, where the community
accumulates species that invest in traits that promote coexistence but 
excludes those that invest in traits promoting exclusion.




% Kisdi2001,Falster2003 papers

% monopolization





% % A Conclusions section is common in AmNat, but not required
% % \section*{Conclusions}

