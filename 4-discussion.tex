
\section*{Discussion}


Here, we showed how coevolution shapes the evolution of traits that govern 
the strength of competition between species. Two types of competition traits
have distinct effects on the rate and outcome of coevolution. Conflict traits
increase the impacts of competition on other species while simultaneously
reducing competition for the evolving species. In contrast, partitioning 
traits simultaneously reduce the strength of competition experienced by the
species evolving the traits and the species with which it interacts. 
The level of investment in conflict and partitioning traits depends on 
whether costs for investing simultaneously in both types of traits are 
additive, sub-additive, or super-additive. If the trade-off between 
partitioning and conflict traits are sub-additive or additive, we would 
expect species to invest in both, whereas super-additive costs should 
result in investment strategies of only partitioning or only conflict traits.
Furthermore, non-heritable variation acts to make the effective trade-off 
more sub-additive; additive and even weakly super-additive trade-offs 
become effectively sub-additive when there is sufficient non-heritable
variation, resulting in simultaneous investment in both trait types. 
Thus, the nature of investment costs, and how they are modified by 
non-heritable variation, determine whether species evolve a mixed 
strategy comprising both partitioning and conflict traits (sub-additive) 
or evolve a specialist strategy of only partitioning or only conflict 
traits (super-additive). In both cases, however, we would expect that
communities should always contain mixed investments, either by all species
showing mixed strategies or by different species specializing on 
different strategies.
Although partitioning and conflict traits should occur simultaneously in
communities, they affect community coevolution in different ways. 
The evolution of conflict traits by a species causes both increases in 
the abundance of the evolving species (by reducing competition it 
experiences) and increases in the per capita strength of competition 
it exerts on other species. Therefore, the evolution of conflict traits 
leads to greater ecological (abundance) and evolutionary (traits) impacts 
on a community than partitioning traits.

Despite the ubiquity of conflict traits, their impacts on competitive
communities are relatively understudied, partly because the question of 
how so many species can coexist in nature is assumed to be answered by 
resource partitioning. However, some empirical evidence has shown that
exclusion-promoting traits, not necessarily niche partitioning, often 
evolve in response to changes in competition 
\citep{Germain2020, Hart2019, Miller2014, Zhao2016}.
Furthermore, recent work on annual plant communities
found that average fitness shows greater phylogenetic signal than niche
partitioning \citep{Godoy2014} and that many commonly measured
functional traits correspond more closely to competitive dominance than
to niche differences \citep{Kraft2015}. These together suggest
that conflict traits that reduce competition for a focal species at the
expense of its competitors might be important for the evolution of
competitive communities. Our results support this hypothesis and show
that trait evolution for entire communities should be driven most
strongly by investments in conflict traits.

Although investors in conflict traits should drive greater investments
in other species, their suppression of competitors' abundances combined
with a cost to investment causes these conflict species themselves to
evolve reduced investment. Conflict-investing species invading a
community should therefore experience evolutionary disarmament after the
initial period of ecological suppression of competitors. On the surface,
this appears to conflict with two major hypotheses related to invasive
species evolution. First, the novel weapons hypothesis predicts
evolution of stronger armament after the invasion because of selection
to suppress native competitors \citep{Callaway2004,
Inderjit2006}. Second, the evolution of increased competitive
ability (EICA) hypothesis predicts evolution of greater competitive
dominance of invaders because they are released from selection due to
natural enemies \citep{Blossey1995}. However, our simulations do
show small evolutionary increases in investments by invaders upon
arrival (fig. \ref{fig:exclusion}E,F), but this is followed by a slower and stronger
decrease after native species are suppressed. This suggests a secondary
stage of invasive species competitive coevolution, where they evolve
reduced suppressive traits when natives are rare enough that the costs
of maintaining those traits are no longer outweighed by the benefits.

These two stages of invader evolution, first to increase investment in
competitive traits and then to decrease investment, may also play out
through space as the invasive species coevolves with natives that are
more abundant at the edge of the invader's range \citep{Miller2020,
Thompson2005a}. Empirical evidence for this comes from garlic
mustard (\emph{Alliaria petiolata}), a widespread Eurasian invader of
North American forest understories. Garlic mustard produces allelopathic
compounds that suppress heterospecific plants but are costly when only
competing with conspecifics \citep{Evans2016}. A number of
studies have shown that its allelopathic effects on North American
competitors decrease with time after invasion and that this pattern is
consistent with evolution via selection for reduced allelopathy
\citep{Bossdorf2004, Evans2016, Huang2018, Lankau2009}. For
costly traits that effectively suppress native species, this secondary
stage of disarmament might be common.

Another effect of evolutionary disarmament by conflict-trait investors
is that these species should have the greatest likelihood of being
excluded from a community. This ``what goes around comes around''
phenomenon makes communities containing many species with conflict
traits eco-evolutionarily transitory. It also means that native species 
that suppress others but have gone through disarmament might be most 
vulnerable to exclusion, despite being more abundant than their 
competitors. Like many eco--evolutionary processes
\citep{Losos2009}, the impermanence of heavy conflict investors
might be most obvious on remote islands. With long periods of
evolutionary disarmament in the absence of new colonizations, repeat
invasions by conflict investors from the competitor-rich mainland to
competitor-poor islands could produce cycles of invasion, dominance,
disarmament, and extinction. This is similar to ``taxon cycles,'' where
species go through repeated periods of greater dispersal to islands,
followed by evolved specialization and range fragmentation once they
establish on islands that ultimately cause them to go extinct
\citep{Losos2009, Wilson1961a}. The evidence for taxon cycles is
mixed \citep{Losos1992, Losos1990, Mayr2001, Taper1992} but
strongest for birds in the West Indies \citep{Ricklefs2002} and
for Melanesian ants \citep{Darwell2020, Economo2012,
Wilson1961a}. One of the main observations made to support taxon
cycles is the transition of recent invaders being abundant and
widespread to older taxa having a greater risk of extinction
\citep{Ricklefs2010}. This is also consistent with serially
invading conflict investors followed by disarmament and may indicate
that this, along with other processes such as coevolution of natural
enemies \citep{Ricklefs2002}, might underlie some of the
episodic extinction--colonization events that exemplify taxon cycles.



% ==============================================================================
% ==============================================================================

\section*{Conclusions}

In trying to understand how trait evolution affects species
coexistence, research most often focuses
only on traits promoting coexistence, such as those
generating resource partitioning. Here, we show that communities are
likely to contain evolutionary investments in both coexistence- and
exclusion-promoting traits, and that the latter should have the greatest
community-wide effects on abundances and investments in competitive
traits. Because traits promoting exclusion suppress the abundance of
competitors, in the long run selection should favor reduced investment
in species possessing these traits. In turn, lower investment
corresponds to a higher risk of later exclusion by an invading species.
Whether coevolution promotes coexistence or exclusion may depend on the
competition traits possessed by a given species. Coevolution may provide
an open door for those that play nice and a revolving door of exclusion
for those that do not.


