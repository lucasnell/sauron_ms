
\section*{Introduction}

% Alternative community states can be described by both the number of species present and by the
% ecologically relevant traits that those species possess. How communities arrive at these states
% is a longstanding question in community ecology
% \citep{Drake:1991bv,Weiher:1999tf,Gleason:1927cj,Clements:1936hw}.
% How community processes shape trait evolution and species filtering is an equally enduring
% question in evolutionary ecology
% \citep{Darwin:1859to,Loeuille:2018cx,Pontarp:2018hv,MacArthur:1964uv,Schluter:2000jz,Muschick:2012ha}.
% Despite the length of time spent on these topics, few generalizations exist \citep{Lawton:1999fj}, at
% least partly because of context-dependent mechanisms \citep{Drake:1991bv} and the complex effects of
% history \citep{Drake:1991bv,Chase:2003ko,Weiher:1999tf}.
%
% Competition has been a particularly well-studied process for shaping communities and trait evolution
% \citep{Simpson:1953wr,Volterra:1928fy,Macarthur:1964kv,Hardin:1960ep,Roughgarden:1976eh,Rosenzweig:1978bj,
% Armstrong:1980id,Hutchinson:1959tq,BrownJr:1956wi,Day:2004db}.
% In many theoretical models, competition strength between two species is inversely proportional to
% the difference in their ecologically relevant trait values
% (\citealp{Abrams:1983jz,Macarthur:1967jf,Volterra:1928fy,Macarthur:1964kv,Rosenzweig:1978bj};
% reviewed in \citealp{Taper:1992kz,Taper:1985ub,Abrams:1986tx,Dayan:2005ub}).
% Models can include this relationship either explicitly \citep[e.g.,][]{Burger:2006tq,Roughgarden:1976eh,Zu:2008uw}
% or implicitly, where trait values represent the ability to extract different resources
% \citep[e.g.,][]{Macarthur:1964kv,Ackermann:2004bb}. In either case, this process can generate or maintain
% trait diversity in a range of forms, including alternative stable states and limit cycles
% \citep{Gilpin:1975gz,Burger:2006tq}.
% In other models, traits change how species perform in competitive contests.
% This can often lead to arms races, where traits cycle or continually increase
% \citep{MaynardSmith:1986tw,Parker:1983io}.
% When \citet{Abrams:1994th} explicitly included population dynamics into a similar model, more complex
% patterns emerged, such as dimorphism and alternative stable states.
%

\textit{\textbf{(I see the paragraph below as the last one in the introduction.)}}

Here, we use a model of evolution in axes affecting competition to describe
the conditions under which multiple-species coexistence is most likely.
We intend for this model to be a general one, so we do not impose a particular 
mechanism on the relationship between axes and competition’s effects;
axes directly affect competitors' per-capita growth rates.
Our model includes only costs, benefits, and non-additive effects of each axis. 
Costs and non-additive effects directly affect the growth rate of the focal species.
Benefits reduce the effects of competition and can either increase or decrease 
competition experienced by other species, depending
on whether evolution is conflicting or non-conflicting.


